\subsection{Strumenti per lo sviluppo}
Abbiamo utilizzato \emph{git} come \emph{Version Control System}, sfruttando ampiamente diverse feature come le branch per lavorare contemporaneamente su più algoritmi ancora incompleti, ed il comando \verb|git bisect| per verificare dove è stato introdotto un certo errore. Come abbiamo anticipato nell'introduzione della Sezione \ref{sec:bispy}, il codice sorgente è caricato su GitHub in un repository pubblico.

Il pacchetto è testato con la libreria \texttt{PyTest} \cite{pytest}, e nel momento in cui stiamo scrivendo ha \emph{code coverage} (che grossolanamente può essere definita come la percentuale di righe di codice coperte da almeno un test) pari a 97\%. Quest'ultima è stata calcolata tramite il tool \emph{coveralls}, che ``esamina'' il codice a richiesta e fornisce la \emph{code coverage}.

Per automatizzare le interazioni con le API di \emph{coveralls} abbiamo utilizzato \emph{GitHub Actions}, un framework che consente di specificare alcune azioni da compiere (la richiesta di revisione a \emph{coveralls}, ad esempio) in risposta ad eventi di vario genere (nel nostro caso il ``push'' di una o più commit sul repository GitHub).
