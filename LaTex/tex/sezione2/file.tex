\section{Algoritmi risolutivi}
In questa sezione esamineremo alcuni algoritmi risolutivi proposti per la risoluzione del problema della determinazione della massima bisimulazione su un grafo. Come risulterà evidente nel seguito viene sfruttata ampiamente l'equivalenza tra bisimulazione massima e RCSP dimostrata nella sezione precedente.

\subsection{Algoritmo di Hopcroft}
Esaminiamo innanzitutto l'algoritmo risolutivo per una versione semplificata del problema della determinazione della bisimulazione massima, ovvero la minimizzazione di un'automa a stati finiti, presentato in \cite{hopcroft} nel 1971. Sebbene questa soluzione non sia generale, ha fornito alcuni spunti fondamentali per l'ideazione di algoritmi risolutivi più completi, che verranno presentati nel seguito del lavoro.

\subsubsection{Alcune nozioni fondamentali}
Innanzitutto definiamo il concetto di \emph{automa}, chiaramente centrale nella descrizione del problema:
\begin{definition}
    Consideriamo i seguenti oggetti:
    \begin{itemize}
        \item Un insieme finito $I$ detto \emph{insieme degli ingressi};
        \item Un insieme finito $S$ detto \emph{insieme degli stati}, ad ogni stato è associata un'unica uscita;
        \item Un insieme finito $F \subseteq S$ detto \emph{insieme degli stati finali};
        \item Una funzione $\delta: S \times I \to S$ detta \emph{funzione di trasferimento}.
    \end{itemize}
    Chiameremo $A = (S,I,\delta,F)$ \emph{automa}. Useremo la notazione $Out(x)$ per indicare l'output corrispondente allo stato $x$.
\end{definition}
Possiamo rappresentare un'automa con una tabella degli stati, in cui ogni riga rappresenta uno stato, ogni colonna un ingresso, ed ogni cella contiene il nuovo stato del sistema quando, nel momento in cui il sistema si trova nello stato corrispondente alla riga, si inserisce l'ingresso corrispondente alla colonna.In alternativa possiamo utilizzare una rappresentazione grafica, in cui ogni stato è descritto da un cerchio contenente il nome dello stato, e le transizioni tra stati sono descritte da frecce, sulle quali viene specificato l'ingresso che ha innescato la transizione.\\
\begin{example}
    Nella Tabella \ref*{fig:tab_automata} è rappresentato un'automa che cambia stato ($A \to B$, $B \to C$) solamente se l'ingresso è ``1'', e lo stato non è ``$C$''. In qualsiasi altro caso lo stato non cambia.
    \begin{table}[ht]
        \centering
        \begin{tabular}{ c | c c }
            \hline
            & 0 & 1\\
            \hline
            a[0] & a & b \\
            b[0] & b & c \\
            c[1] & c & c \\
            \hline
          \end{tabular}
        \caption{Rappresentazione tabellare di un'automa}
        \label{fig:tab_automata}
    \end{table}
    \label{exa:automata_tab}
\end{example}
\begin{example}
    Nella Figura \ref*{fig:automata} è rappresentato lo stesso automa dell'Esempio \ref*{exa:automata_tab}.
    \begin{figure}[hb]
        \centering
        \begin{tikzpicture}[->,>=stealth',shorten >=1pt,auto,node distance=3.5cm,
            scale = 1,transform shape]

            \node[state] (a) {$a[0]$};
            \node[state] (b) [right of=a] {$b[0]$};
            \node[state] (c) [right of=b] {$c[1]$};

            \path (a) edge              node {$1$} (b)
                  (b) edge              node {$1$} (c)
                  (a) edge [loop above]             node {$0$} (a)
                  (b) edge [loop above]             node {$0$} (b)
                  (c) edge [loop above]              node {$0/1$} (c);
        \end{tikzpicture}
        \caption{Rappresentazione grafica di un'automa}
        \label{fig:automata}
    \end{figure}
\end{example}
In alcuni automi è possibile individuare stati che ``si comportano in modo simile''. Informalmente, questi stati reagiscono allo stesso modo all'ingresso degli stessi input. Diamo un esempio di questa situazione:
\begin{example}
    \begin{figure}[b]
        \centering
        \begin{tikzpicture}[->,>=stealth',shorten >=1pt,auto,node distance=3.5cm,
            scale = 1,transform shape]

            \node[state] (a) {$a[0]$};
            \node[state] (b) [right of=a] {$b[1]$};
            \node[state] (c) [right of=b] {$c[1]$};

            \path (a) edge              node {$1$} (b)
                  (b) edge [bend right]             node {$1$} (c)
                  (c) edge [bend right]             node {$1$} (b)
                  (a) edge [loop above]             node {$0$} (a)
                  (b) edge [loop above]             node {$0$} (b)
                  (c) edge [loop above]             node {$0$} (c);
        \end{tikzpicture}
        \caption{Automa contenente stati equivalenti}
        \label{fig:automata_eq}
    \end{figure}
    Consideriamo gli stati dell'automa rappresentato graficamente nella Figura \ref*{fig:automata_eq}. Supponiamo di accorpare gli stati $B,C$ in un unico stato, che chiamiamo $B'$. Se ci interessiamo solamente alla sequenza di output ed all'eventuale raggiungimento di uno stato finale, il nuovo automa risulta indistinguibile dal primo.
\end{example}
Proponiamo la definizione formale di equivalenza tra stati che viene utilizzata in \cite{hopcroft}. Nel seguito ne dedurremo un'altra, che consente di stabilire un parallelo con gli argomenti trattati nella Sezione precedente.
\begin{definition}
    \label{def:equivalent_states}
    Sia $I^*$ l'insieme di tutte le sequenze di input di lunghezza finita. Sia $\delta^* : S \times I^*$ la funzione di transizione ``iterata''. Diremo che due stati $x,y$ sono equivalenti (con la notazione $x \sim y$) se e soltanto se valgono congiuntamente le seguenti condizioni:
    \begin{enumerate}
        \item $Out(x) = Out(y)$;
        \item $\forall \widetilde{i} \in I^*,\,\, \delta^*(x,\widetilde{i}) \in F \iff \delta^*(y,\widetilde{i}) \in F$.
    \end{enumerate}
\end{definition}
\accente evidente che individuare gli stati equivalenti consente di minimizzare il sistema, preservando al tempo stesso i risultati ottenuti. In questo lavoro non illustreremo cosa comporta questa definizione, e perchè è importante che per qualsiasi stringa di lunghezza finita si giunga ad uno stato finale partendo da due stati supposti equivalenti.
\begin{observation}
    La relazione ``$\sim$'' così definita è una relazione di equivalenza sull'insieme degli stati di un'automa.
\end{observation}
La seguente osservazione sarà utile per formulare il problema con la terminologia esposta nella Sezione \ref*{sec:rscp}.
\begin{observation}
    \label{obs:equivalent_states}
    Due stati $x,y$ sono equivalenti nel senso della definizione \ref*{def:equivalent_states} se e solo se
    \begin{gather}
        \forall i \in I, \quad \delta(x,i) \sim \delta(y,i)
    \end{gather}
    e $Out(x) = Out(y)$.
\end{observation}
\begin{proof2}
    Supponiamo che $\exists i \in I : \delta(x,i) \not\sim \delta(y,i)$. Allora, ad esempio:
    \begin{gather*}
        \exists i^* \in I^* : \delta^*(\delta(x,i),i^*) = \delta^*(x,ii^*) \in F, \,\, \delta^*(\delta(y,i),i^*) = \delta^*(y,ii^*) \notin F
    \end{gather*}
    Quindi, per l'esistenza della string $ii^*$ si ha $x \not\sim y$. La dimostrazione è speculare se $\exists i^* \in I^* : \delta^*(x,ii^*) \notin F, \delta^*(y,ii^*) \in F$.\\
    Ora supponiamo che valga la (1). E' chiaro che $\forall i^* \in I^*, \forall i \in I$ si ha che
    \begin{gather*}
        \delta^*(x,ii^*) \sim \delta^*(y,ii^*)
    \end{gather*}
    e quindi $x \sim y$.
\end{proof2}
\begin{observation}
    La relazione ``$\sim$'' con la formulazione dell'Osservazione \ref*{obs:equivalent_states} è una bisimulazione sull'insieme degli stati di un'automa, se si considera la relazione binaria $\displaystyle \to \,\,\coloneqq \bigcup_{i \in I, x \in S} \{(x,\delta(x,i))\}$.
\end{observation}
\begin{proof2}
    Supponiamo che $x \sim y$. Sia $x \to x'$, cioè $\exists i \in I : x' = \delta(x,i)$. Sia $y' = \delta(y,i) \implies y \to y'$. Allora, per l'Osservazione \ref*{obs:equivalent_states}, $x' \sim y'$.\\
    Chiaramente lo stesso argomento vale in modo speculare.
\end{proof2}
Osserviamo che nella definizione di ``$\to$'' si perde un'informazione importante, cioè il fatto che per $i \in I$ fissato $\delta(x)$ è una funzione, ovvero l'insieme immagine di ogni $x \in S$ ha cardinalità 1. L'algoritmo di Hopcroft sfrutta questa proprietà nel procedimento che consente di migliorare l'algoritmo banale che verrà discusso nel seguito del lavoro.\\
Possiamo definire la \emph{minimizzazione per stati equivalenti} di un'automa a stati finiti:
\begin{definition}\label{def:minim_eq_states}
    Sia $A = (S,I,\delta,F)$ un'automa. Chiameremo \emph{minimizzazione per stati equivalenti} l'automa $A = (\widetilde{S}, I, \widetilde{\delta}, \widetilde{F})$ dove
    \begin{itemize}
        \item $\widetilde{S} =$ ``l'insieme delle classi di equivalenza di $\sim$ su $S$'';
        \item $\widetilde{\delta} : (\widetilde{S} \times I) \to \widetilde{S}, \quad \delta(i, x) = y \implies \widetilde{\delta}(i,\widetilde{x}) = \widetilde{y}$, dove $x,y \in S, i \in I$, e $\widetilde{x}, \widetilde{y} \in \widetilde{S}$ sono rispettivamente le classi di equivalenza di ``$\sim$'' a cui appartengono $x,y$;
        \item $\widetilde{F} =$ ``l'insieme delle classi di equivalenza di $\sim$ su $F$'';
    \end{itemize}
\end{definition}
Concludiamo l'esposizione delle nozioni preliminari con la seguente osservazione:
\begin{observation}
    Sia $A$ un'automa a stati finiti. La minimizzazione per stati equivalenti dell'automa è l'automa avente per stati i blocchi della partizione più grossolana stabile rispetto alle funzioni $\delta_i : S \to S, \forall i \in I$.
\end{observation}
\begin{proof2}
    Dimostriamo che la minimizzazione proposta nella Definizione \ref*{def:minim_eq_states} è una partizione stabile rispetto alle funzioni $\delta_i$. Supponiamo per assurdo che $\exists i \in I : \widetilde{S}$ non è stabile rispetto a $\delta_i$. Quindi $\exists S_1, S_2 \in S :$
    \begin{gather*}
        S_1 \not\subseteq \delta_i^{-1}(S_2) \,\, \land \,\, S_1 \,\cap \,\delta_i^{-1}(S_2) \neq \emptyset
    \end{gather*}
    La prima porzione dell'espressione implica che $\exists s \in S_1 : \delta_i(s) \not\in S_2$. Ma questo è chiaramente in contrasto con l'Osservazione \ref*{obs:equivalent_states}, perchè $\forall (u,v) \in S_1 \times S_1$ deve valere $\delta_i(u) \sim \delta_i(v)$, mentre è evidente che, poichè $S_1 \,\cap\, \delta_i^{-1}(S_2) \neq \emptyset$, c'è almeno una coppia che non soddisfa questa condizione.\\
    Resta da dimostrare che la partizione proposta $\widetilde{S}$ è la più grossolana stabile rispetto alle funzioni $\delta_i : S \to S, \forall i \in I$... % TODO: dimostra
\end{proof2}

\subsubsection{L'algoritmo}
Innanzitutto, con i risultati della sezione precedente, possiamo progettare il seguente algoritmo banale:\\
\begin{algorithm}[H]
    \SetAlgoLined
    \KwData{$S,I,\delta$}
    \tcp*[h]{Partizione iniziale contenente un unico blocco}
    \nl$\widetilde{S} \coloneqq \{\{s_1, \dots, s_n\}\}$\;
    \tcc*[h]{Separiamo gli stati non equivalenti in base all'output}\\
    $\widetilde{S} = $ PartizionaPerOutput($\widetilde{S}$)\;\label{alg:hop_banale_partiziona_output}
    \For{$i \in I$} {
        \label{alg:hop_banale_for_ingressi}
        \For{$NS = $ BlocchiNonStabili(S,$\delta_i$)), $NS \neq \emptyset$}{
            \label{alg:hop_banale_for_ns}
            \tcc*[h]{Estraiamo casualmente una coppia di blocchi non stabili}\\
            $(X,Y) = NS[0]$\;
            $X_1 \coloneqq X - \delta_i^{-1}({Y})$\;
            $X_2 \coloneqq X \cap \delta_i^{-1}({Y})$\;
            \tcp*[h]{Aggiorniamo $S$}\\
            $\widetilde{S} = (\widetilde{S} - \{X\}) \cup \{X_1,X_2\}$\;
        }
    }
    \Return{$S$}
    \caption{Procedimento banale per la minimizzazzione}
\end{algorithm}
L'algoritmo termina, perchè la condizione del ciclo diventa sicuramente falsa quando la partizione $S$ è composta da $n$ blocchi, uno per ogni stato, ed ad ogni iterazione un blocco viene diviso in due blocchi distinti e non vuoti. \\
Inoltre la risposta fornita è corretta, perchè l'algoritmo si ferma appena viene trovata una partizione stabile. Poichè ad ogni iterazione del ciclo definito nella Riga \ref*{alg:hop_banale_for_ns} il numero di blocchi aumenta di 1, il risultato è la partizione stabile più grossolana rispetto alle funzioni $\delta_i$. Osserviamo inoltre che, se la partizione è stabile rispetto a $\delta_i$ dopo una certa iterazione del ciclo definito nella Riga \ref*{alg:hop_banale_for_ingressi}, allora resta stabile fino alla fine dell'esecuzione.\\
Consideriamo la complessità dell'algoritmo:
\begin{itemize}
    \item La Riga \ref*{alg:hop_banale_partiziona_output} ha complessità $\Theta(|S|)$;
    \item Il ciclo della Riga \ref*{alg:hop_banale_for_ingressi} viene eseguito $\Theta(|I|)$ volte;
    \item La funzione ``BlocchiNonStabili'' ha complessità $O(|S|^2)$;
    \item Il ciclo nella Riga \ref*{alg:hop_banale_for_ns} viene eseguito $O(|S|)$ volte;
    \item Il contenuto del ciclo nella Riga \ref*{alg:hop_banale_for_ns} ha complessità $O(|S|)$ (con gli opportuni accorgimenti).
\end{itemize}
Quindi la complessità dell'algoritmo è $\Theta|I|(O(|S|^2 + O(|S|) * O(|S|))) = \Theta(I)O(|S|^2)$. Se consideriamo costante il numero di ingressi, la complessità diventa $O(|S|^2)$.
