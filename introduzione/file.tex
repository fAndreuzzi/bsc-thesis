\section*{Introduzione}
\addcontentsline{toc}{section}{Introduzione}

Tramite i \emph{grafi} è possibile formulare modelli informatici di sistemi complessi di varia natura, come ad esempio \emph{social network} \cite{twitter}, reti fisiche \cite{electric} o logistiche \cite{supply}. In tali modelli risulta difficoltoso individuare una nozione univoca di equivalenza tra i nodi che compongono il sistema, in quanto diverse definizioni catturano differenti aspetti del problema. Una nozione di equivalenza può essere utilizzata per condurre analisi sul grafo (dividere i nodi in un partizionamento che ne catturi una qualche caratteristica), o per ottimizzare lo spazio occupato dal modello eliminando informazioni ridondanti.

In questo lavoro prenderemo in considerazione una possibile nozione di equivalenza, la \emph{bisimulazione} (secondo la definizione standard); in particolare ci concentreremo sull'aspetto informatico del problema, e presenteremo alcuni algoritmi che consentono di calcolare in modo efficiente un partizionamento del sistema rispetto a questa nozione di equivalenza, ovvero di determinare quali coppie di nodi sono tra loro equivalenti.

Presenteremo inoltre un pacchetto in cui sono stati implementati gli algoritmi per il calcolo della massima bisimulazione di grafi diretti. Si tratta di uno strumento inedito nell'ambiente Python \emph{open source}, in quanto non ci è stato possibile trovare altri progetti che trattassero il medesimo argomento.

Il lavoro è organizzato in questo modo:
\begin{itemize}
    \item Nella Sezione \ref{sec:base} forniremo innanzitutto alcune nozioni fondamentali nell' ambito di teoria dei grafi, teoria degli insiemi e relazioni binarie, per poi introdurre le definizioni di bisimulazione e massima bisimulazione, insieme ad alcuni risultati che utilizzeremo per le analisi successive;
    \item Nella Sezione \ref{sec:algs} analizzeremo alcuni algoritmi efficienti per il problema della massima bisimulazione, e ne dimostreremo in modo dettagliato la correttezza e la complessità computazionale;
    \item Nella Sezione \ref{sec:bispy} presenteremo il pacchetto \texttt{BisPy}.
\end{itemize}
