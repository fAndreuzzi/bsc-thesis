\section{Introduzione}
La \emph{teoria dei grafi} è uno strumento potente per formulare modelli informatici di sistemi complessi caratterizzati da strutture a ``nodi e relazioni'', come ad esempio \emph{social network} o reti fisiche di varia natura. Se tali modelli riproducono con efficacia le caratteristiche del sistema originale, viene trasferita anche la difficoltà naturale di individuare una qualche definizione di ``equivalenza'' tra i nodi.

In questo lavoro presenteremo una possibile risposta a questo problema, la \emph{bisimulazione} (secondo la definizione standard), e ci concentreremo in particolare sulla \emph{massima bisimulazione}, che come vedremo è anche una \emph{relazione di equivalenza}. Presenteremo alcuni algoritmi che consentono di calcolare in modo efficiente un partizionamente del sistema rispetto a questa nozione di equivalenza, ovvero di determinare quali coppie di nodi sono tra loro equivalenti.

L'elaborato è organizzato in questo modo: nella Sezione \ref{sec:base} forniremo innanzitutto alcune nozioni fondamentali nell'ambito di teoria dei grafi, teoria degli insiemi e delle relazioni binarie, per poi introdurre le definizioni di bisimulazione e massima bisimulazione, insieme ad alcuni risultati che utilizzeremo per le analisi successive; nella Sezione \ref{sec:algs} analizzeremo alcuni algoritmi efficienti per la risoluzione del problema della massima bisimulazione, e ne dimostreremo in modo dettagliato la correttezza e la complessità computazionale; infine nella Sezione \ref{sec:bispy} presenteremo brevemente il package Python \texttt{BisPy} sviluppato in parallelo a questo elaborato, in cui abbiamo implementato gli algoritmi che abbiamo preso in considerazione.
