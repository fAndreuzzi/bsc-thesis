\usepackage{amsthm}
\usepackage{amssymb}
\usepackage{amsmath}
\usepackage{cite}
\usepackage{hyperref}
%\usepackage[hidelinks]{hyperref}
\usepackage{mathtools}
\usepackage{subcaption}
\usepackage{tikz}
\usepackage{tikzsymbols}
\usepackage{pgfplots}
\usepackage{wrapfig}
\usepackage{xspace}
\usepackage{float}
\usepackage{caption}
\usepackage[nottoc]{tocbibind}
\usepackage[english]{babel}
\usepackage[utf8x]{inputenc}
\usepackage{venndiagram}
\usepackage{xspace}
\usepackage{centernot}
\usepackage{afterpage}
\usepackage[linesnumbered,ruled,lined,boxed,vlined,noend]{algorithm2e}
\usepackage[[deletedmarkup=xout, commentmarkup=footnote]{changes}
\usepackage{mathtools}

\pgfplotsset{compat=1.11,
                /pgfplots/ybar legend/.style={
                /pgfplots/legend image code/.code={%
                \draw[##1,/tikz/.cd,yshift=-0.25em]
                    (0cm,0cm) rectangle (3pt,0.8em);},
                },
}

\def\blankpage{%
      \clearpage%
      \thispagestyle{empty}%
      \addtocounter{page}{-1}%
      \null%
      \clearpage}

\numberwithin{equation}{section}

\newsavebox{\largestimage}

\newcommand{\accente}{\`E }

\DeclareMathOperator*{\argmax}{arg\,max}
\DeclareMathOperator*{\argmin}{arg\,min}

\DeclarePairedDelimiter\ceil{\lceil}{\rceil}

\newcommand{\funcname}[1]{\texttt{\textup{#1}}}
\newcommand{\splitfunc}{\funcname{split}\xspace}
\newcommand{\rankfunc}{\funcname{rank}\xspace}
\newcommand{\rscp}{\textup{RSCP}\xspace}
\newcommand{\prefunc}{\funcname{Pre}\xspace}
\newcommand{\succfunc}{\funcname{Succ}\xspace}
\newcommand{\wf}{\funcname{WF}\xspace}
\newcommand{\nwf}{\funcname{NWF}\xspace}

\makeatletter
\renewcommand{\boxed}[1]{\text{\fboxsep=.2em\fbox{\m@th$\displaystyle#1$}}}
\makeatother

\newcommand\restr[2]{{% we make the whole thing an ordinary symbol
  \left.\kern-\nulldelimiterspace % automatically resize the bar with \right
  #1 % the function
  \vphantom{\big|} % pretend it's a little taller at normal size
  \right|_{#2} % this is the delimiter
  }}

\newcommand\sccto{\stackrel{\mathclap{\normalfont\mbox{\tiny{SCC}}}}{\to}}
\newcommand\superscc{\scalebox{.5}{SCC}}

\definechangesauthor[name=Alberto Casagrande, color=red]{ac}

\usetikzlibrary{arrows,automata,patterns,shapes,snakes,arrows.meta}

\graphicspath{ {imgs/} }

\captionsetup[figure]{name=Figura}
\captionsetup[table]{name=Tabella}
\renewcommand{\algorithmcfname}{Algoritmo}
\renewcommand\labelitemi{--}

\SetArgSty{textnormal}

\theoremstyle{definition}
\newtheorem{definition}{Definizione}[section]
\newtheorem{example}{Esempio}[section]
\newtheorem{axiom}{Assioma}[section]
\newtheorem*{axiom*}{Assioma}
\newtheorem{observation}{Osservazione}[section]
\newtheorem*{observation*}{Osservazione}

\theoremstyle{plain}
\newtheorem{proposition}{Proposizione}[section]
\newtheorem{lemma}{Lemma}[section]
\newtheorem{theorem}{Teorema}[section]
\newtheorem{corollary}{Corollario}[section]

\newcommand{\notimplies}{\mathrel{{\ooalign{\hidewidth$\not\phantom{=}$\hidewidth\cr$\implies$}}}}

\newenvironment{proof2}
{
  \begin{proof}[Dimostrazione]
}
{\end{proof}}
