\subsection{Relazioni Binarie}
Riportiamo la definizione di \emph{relazione binaria} su uno o due insiemi, che sarà utile per definire formalmente il concetto di \emph{grafo}, fondamentale all'interno di questo elaborato:
\begin{definition}
    Una \emph{relazione binaria} su $A,B$ è un sottoinsieme del prodotto cartesiano $A \times B$.\\
    Una \emph{relazione binaria} su $A$ è un sottoinsieme del prodotto cartesiano $A \times A$.\\
	Se $R$ mette in relazione $u,v$, cioè $(u,v) \in R$, si usa la notazione $u R v$.
\end{definition}
\begin{definition}
    L' \emph{insieme immagine} di un elemento $x$ dell'insieme $A$ attraverso la relazione $R$ è l'insieme $R(x) = \{y \in B \mid x R y\}$.
\end{definition}
Alcune relazioni binarie mostrano proprietà fondamentali, che presentiamo nella definizione seguente:
\begin{definition}
    Sia $R$ una relazione binaria su $A$. Siano $x,y,z$ qualsiasi appartenenti ad $A$. Allora $R$ è:
    \begin{itemize}
        \item \emph{Riflessiva} se $x R x$;
        \item \emph{Simmetrica} se $x R y \implies y R x$;
        \item \emph{Transitiva} se $(x R y \land y R z) \implies x R z$.
    \end{itemize}
\end{definition}
\begin{example}
    La relazione ``$\leq$'' sui naturali è riflessiva e transitiva, ma non simmetrica. La relazione ``$=$'' ($a = b \iff $``$a,b$ sono lo stesso numero'') sui naturali è simmetrica, riflessiva e transitiva.
\end{example}
\begin{definition}
    Una \emph{relazione di equivalenza} su un insieme $A$ è una relazione binaria riflessiva, simmetrica e transitiva. Si vede facilmente che questo genere di relazione partiziona $A$ in \emph{classi di equivalenza}, ovvero sottoinsiemi disgiunti di $A$ all'interno dei quali tutte le coppie di elementi sono in relazione.
\end{definition}
Data una relazione di equivalenza $R$ su un insieme $A$, si usa la notazione ``$[a]_R$'' per indicare la classe di equivalenza di $R$ a cui appartiene $a$. Inoltre si usa la notazione ``$A/R$'', che si legge ``\emph{quoziente} di $A$ rispetto a $R$'', per denotare l'insieme delle classi di equivalenza di $R$ su $A$.\\
In alcune situazioni risulta conveniente definire la più piccola relazione (cioè quella che mette in relazione il minor numero possibile di coppie) che dispone di una certa proprietà, e che contiene una relazione binaria di partenza. Una relazione costruita in questo modo è una ``\emph{chiusura}'':
\begin{definition}
	Sia $R$ una relazione binaria su $A$. Le seguenti relazioni sono chiusure di $R$:
    \begin{itemize}
        \item \emph{Riflessiva}: $R_r = R \cup \{(x,x) \mid x \in A\}$;
        \item \emph{Simmetrica}: $R_s = R \cup \{(y,x) \mid x R y\}$;
        \item \emph{Transitiva}: $R_t = R \cup \{(x,z) \mid \exists y \in A,\, x R y \land y R z\}$.
    \end{itemize}
\end{definition}
\begin{example}
    La chiusura riflessiva della relazione ``$<$'' (minore stretto) è la relazione ``$\leq$''.
\end{example}
Nel seguito useremo ampiamente la definizione seguente:
\begin{definition}
    Sia $R$ una relazione binaria su $A \times B$. La \emph{contro-immagine} di un elemento $y \in B$ rispetto ad $R$ è l'insieme $R^{-1}(y) = \{x \in A \mid x R y\}$.\\
    Più in generale, la \emph{funzione inversa} di $R$ è la funzione $R^{-1} : \mathcal{P}(B) \to \mathcal{P}(A)$ (dove ``$\mathcal{P}$'' denota l'insieme delle parti) che associa ad un sottoinsieme di $B$ tutti gli $x \in A$ tali che vale $x R y$ per almeno un $y$ del sottoinsieme.
\end{definition}
Adottiamo infine la notazione ``$|A|$'' per indicare la \emph{cardinalità} dell'insieme $A$. In modo analogo, data una relazione binaria $R$, $|R|$ è il numero delle coppie messe in relazione da $R$.
