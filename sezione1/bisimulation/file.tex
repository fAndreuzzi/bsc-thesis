\subsection{Bisimulazione}
In questa sezione introdurremo la definizione di bisimulazione ed alcune proprietà immediate. In seguito esamineremo la relazione tra la teoria degli insiemi e la bisimulazione.
\subsubsection{Definizione e risultati generali}
\begin{definition}
    Siano $G_1 = (V_1,E_1), G_2 = (V_2,E_2)$ due grafi diretti. Una relazione binaria $R: V_1 \times V_2$ è una \emph{bisimulazione} su $G_1, G_2$ se $\forall a \in V_1, b \in V_2$ valgono congiuntamente le seguenti proprietà:
    \begin{itemize}
        \item $a R b, a E_1 a' \implies \exists b' \in V_2 \mid (a' R b' \land b E_2 b')$
        \item $a R b, b E_2 b' \implies \exists a' \in V_1 \mid (a' R b' \land a E_1 a')$
    \end{itemize}
    Analogamente si definisce una bisimulazione su un unico grafo diretto $G$, ponendo $G_1 = G_2 = G$.
\end{definition}
L'esistenza di una bisimulazione tra due grafi, come vedremo, è un'informazione rilevante se siamo interessati agli insiemi rappresentati dai grafi considerati:
\begin{definition}
    Siano $G_1 = (V_1,E_1), G_2 = (V_2,E_2)$ due grafi. Essi sono \emph{bisimili} se esiste una bisimulazione su $G_1, G_2$.\\
    Due APG $(G_1, v_1), (G_2, v_2)$ sono \emph{bisimili} se $G_1, G_2$ sono bisimili e vale $v_1 R v_2$ per almeno una bisimulazione $R$ su $G_1, G_2$.
\end{definition}
\begin{observation}
    Una bisimulazione può non essere riflessiva, simmetrica, nè transitiva.
\end{observation}
\begin{example}
    La relazione $a R b \iff$ ``$a,b$ sono lo stesso nodo'' su un grafo qualsiasi è una bisimulazione riflessiva, simmetrica e transitiva.\\
    La relazione $R = \emptyset$ è una bisimulazione su un grafo qualsiasi, ma non è riflessiva.\\
    La relazione $R = \{(a,a),(b,b),(c,c),(d,d),(a,b),(b,c),(c,d)\}$ sul grafo $G = (\{a,b,c,d\}, \{(a,b),(b,c),(c,d),(d,d)\})$ è una bisimulazione, ed è solamente riflessiva.
\end{example}
Dalla definizione di bisimulazione possiamo dedurre una proprietà interessante delle chiusure:
\begin{theorem}
    Sia $R$ una bisimulazione sul grafo diretto $G$. La sua chiusura riflessiva, simmetrica o transitiva è ancora una bisimulazione su $G$.
\end{theorem}
\begin{proof2}
    Consideriamo separatamente le tre relazioni $R_r, R_s, R_t$, rispettivamente la chiusura riflessiva, simmetrica e transitiva:
    \begin{itemize}
        \item $R_r$: Per definizione $R \subset R_r$, quindi è sufficiente dimostrare che $R_r$ è una bisimulazione quando gli argomenti $u,v \in V$ non sono distinti.\\
        Sia $u \in V$. Chiaramente per definizione di $R_r$ si ha $u R_r u$. Se $\exists u' \in V \mid u E u'$ allora (sempre per definizione di $R_r$) si ha $u' R_r u'$.
        \item $R_s$: Per definizione $R \subset R_s$, quindi è sufficiente dimostrare che $R_s$ è una bisimulazione quando per $u,v \in V$ si ha $u R v$ ma non $v R u$.\\
        Sia $(u,v) \in V\times V$. Allora:
        \begin{center}
            $u R_s v \implies u R v \lor v R u$
        \end{center}
        Supponiamo ad esempio che $v R u$.
        \begin{align*}
            &\implies \forall v' \in V\mid (v E v') \,\,\exists u' \in V, (u E u' \land v' R u')\\
            &\implies u' R_s v'
        \end{align*}
        e
        \begin{align*}
            &\implies \forall u' \in V\mid (u E u') \,\,\exists v' \in V, (v E v' \land v' R u')\\
            &\implies u' R_s v'
        \end{align*}
        cioè sono dimostrate le due condizioni caratteristiche della bisimulazione.\\
        La dimostrazione è analoga se $u R v$.
        \item $R_t$: Per definizione $b \subset R_t$, quindi è sufficiente dimostrare che $R_t$ è una bisimulazione quando per gli argomenti $u,v,z \in V$ si ha $u R v$, $v R z$ ma non $u R z$.\\
        Sia $(u,v,z) \in V\times V\times V$ con questa proprietà. Allora:
        \begin{gather*}
            \forall u' \in V\mid u E u' \,\, \exists v' \in V, (v E v' \land u' R v')
        \end{gather*}
        Inoltre $\exists z' \mid z E z' \land v' R z'$.\\
        Riordinando si ha $u' R v', v' R z'$. Allora per definizione di $b_t, \, u' R_t z'$.\\
        In modo speculare si ottiene la seconda condizione caratteristica della bisimulazione.
    \end{itemize}
    \vspace*{-0.75cm}
\end{proof2}
Da questa proposizione si deduce il seguente corollario, che risulta dall'applicazione iterata delle tre chiusure viste in precedenza:
\begin{corollary}
    Ad ogni bisimulazione $R$ è possibile associare una bisimulazione $\widetilde{R} : R \subset \widetilde{R} \,\,\land\,\, \widetilde{R}$ è una relazione di equivalenza.
    \label{cor:bisimulation_eqrel}
\end{corollary}
Concludiamo la sezione relativa ai risultati generali sulla bisimulazione con la seguente proposizione, che sarà utile nel seguito:
\begin{proposition}
    Siano $R_1, R_2$ due bisimulazioni su $G_1, G_2$. Allora $R = R_1 \cup R_2$ è ancora una bisimulazione.
    \label{obs:bisimulation_union}
\end{proposition}
\begin{proof2}
    Siano $u,v \mid u R v$. Sia $u' \mid u E u'$. Allora deve essere $u R_1 v \lor u R_2 v$. Ma quindi $\exists v' \mid (v E v' \land u' R_{1|2} v')$.
\end{proof2}

\subsubsection{Bisimulazione massima}
\label{sec:bisi_max}
Definiamo ora il concetto di \emph{bisimulazione massima}, che gioca un ruolo chiave nella risoluzione dei problemi considerati in questo elaborato:
\begin{definition}
    Una bisimulazione $R_M$ su $G_1, G_2$ è \emph{massima} se per ogni bisimulazione $R$ su $G_1,G_2 \,\, u R v \implies u R_M v$.
\end{definition}
\begin{observation}
    Sia $R_M$ la bisimulazione massima su un grafo diretto $G$. Allora per qualsiasi altra bisimulazione $R$ su $G$ si ha $|R_M| \geq |R|$.
\end{observation}
Naturalmente la bisimulazione massima dipende dai due grafi presi in esame. Possiamo dedurre alcune caratteristiche banali:
\begin{proposition}
    Valgono le seguenti proprietà:
    \begin{enumerate}
        \item La bisimulazione massima su due grafi $G_1,G_2$ è unica;
        \item La bisimulazione massima è una relazione di equivalenza.
    \end{enumerate}
    \vspace*{-0.3cm}
    \label{prop:bisi_max_equi}
\end{proposition}
\begin{proof2}
    Le proprietà seguono dal Corollario \ref{cor:bisimulation_eqrel} e dall'Osservazione \ref{obs:bisimulation_union}:
    \begin{enumerate}
        \item Supponiamo per assurdo che esistano due bisimulazioni massime $R_{M_1}, R_{M_2}$. La loro unione è ancora una bisimulazione, che è "più massima" delle supposte bisimulazioni massime.
        \item Se per assurdo la bisimulazione massima non fosse una relazione di equivalenza, potremmo considerare la sua chiusura riflessiva, simmetrica e transitiva, che avrebbe cardinalità maggiore o uguale, e sarebbe per costruzione una relazione di equivalenza.
    \end{enumerate}
    \vspace*{-0.7cm}
\end{proof2}
Naturalmente il concetto di \emph{bisimulazione massima} può essere definito anche su unico grafo diretto $G$. Questo caso si rivelerà di grande interesse nel seguito. Per ora dimostriamo il seguente risultato:
\begin{theorem}
    Sia $G$ un grafo diretto finito. Allora esiste la bisimulazione massima su $G$.
\end{theorem}
\begin{proof2}
    Può esistere solamente un numero finito di relazioni binarie su $G$, e questo numero fornisce un limite superiore al numero massimo di bisimulazioni su $G$.
    Allora possiamo considerare l'unione di questo numero finito di bisimulazioni, che sarà chiaramente la bisimulazione massima.
\end{proof2}

\subsubsection{Interpretazione insiemistica della bisimulazione}
Mostriamo ora una conseguenza diretta dell'Assioma di Estensionalità e di AFA:
\begin{theorem}
    Due APG rappresentano lo stesso insieme $\iff$ sono bisimili.
    \label{theo:bisi_iff_eqsets}
\end{theorem}
\begin{proof2}
    Siano $G_A = (V_A, E_A), G_B = (V_B, E_B)$. Dimostriamo separatamente le due implicazioni:
    \begin{enumerate}
        \item[$(\implies)$] Osserviamo innanzitutto che la relazione binaria ``$\equiv$'' su $V_A, V_B$ definita come segue:
              \begin{gather*}
                  a \equiv b \iff ``\text{le decorazioni di } A,B \text{ associano ad } a,b \text{ lo stesso insieme}''
              \end{gather*}
              è una bisimulazione sui grafi $G_A, G_B$.\\
              Chiaramente se $a \equiv b, a E a'$ si ha:
              \begin{itemize}
                  \item $a' \in a$, associando ad $a,a'$ gli insiemi che rappresentano secondo la decorazione (l'unica) considerata;
                  \item $a,b$ rappresentano lo stesso insieme.
              \end{itemize}
              Quindi per l'Assioma di Estensionalità $\exists b' \in b\mid b' = a'$, cioè $b E b'$ e $a' \equiv b'$. Si procede specularmente per la seconda condizione caratteristica della bisimulazione.\\
              La relazione ``$\equiv$'' è una bisimulazione sugli APG $G_A,G_B$ quando si assume per ipotesi che rappresentino lo stesso insieme.
        \item[$(\impliedby)$] Sia $R$ una bisimulazione su $G_A,G_B$. Consideriamo la decorazione $d_A$ (l'unica) di $G_A$. Vogliamo estrapolarne una decorazione per $G_B$, e dimostrare che i due grafi con le rispettive decorazioni sono due immagini dello stesso insieme. Dalla possibilità di operare questo procedimento, dall'Assioma di Estensionalità e da AFA potremo dedurre l'uguaglianza degli insiemi rappresentati.\\
              Definiamo la decorazione $d_B$ come segue:
              \begin{gather*}
                  d_B(v) = d_A(u), \text{con $u$ nodo di $G_A$} \mid u R v
              \end{gather*}
              \begin{observation*}
                  Per ogni nodo $v$ di $B$ deve esistere almeno un nodo $u$ di $G_A \mid u R v$ perchè si suppone che i due APG siano bisimili.
              \end{observation*}
              Dimostriamo che $d_B$ è una decorazione di $G_B$. In altre parole vogliamo dimostrare che per ogni nodo $v$ di $G_B$ l'insieme $d_B(v)$ contiene tutti e soli gli insiemi $d_B(v')$ dove $v'$ è figlio di $v$.
              \begin{itemize}
                  \item Supponiamo per assurdo che tra i figli di $v$ ``manchi" il nodo corrispondente ad un elemento $X \in d_B(v)$. Poichè la decorazione di $G_A$ è ben definita,il nodo $u$ di $G_A \mid u R v$ deve avere un figlio corrispondente a $X$, che chiameremo $u'$. Ma $u R b, u E u' \implies \exists v' \mid v E v', u' R v'$. Cioè $d_B(v') = X$. Dunque il nodo mancante è stato identificato.
                  \item Supponiamo per assurdo che tra i figli di $v$ ci sia un nodo ``in più'', ovvero un nodo $v' \mid v E v', d_B(v') = Y$ con $Y \notin d(v)$. Ma allora, sempre considerando $u$ il nodo di $G_A \mid u R v$, dovrebbe esistere un $u' \mid u E u', u' R v'$, cioè $d_A(u') = d_B(v') = Y$.
                  Ma allora $Y \in d_A(u) = d_B(v)$, deduzione che è chiaramente in contrasto con l'ipotesi.
              \end{itemize}
              \accente possibile che ci siano due nodi di $a_1, a_2$ di $G_A$ ed un nodo $b$ di $G_B$ per cui vale $a_1 R b \land a_2 R b$. In questo caso la decorazione definita è ambigua. Per questo motivo correggiamo la formulazione di $d_B$ come segue:
              \begin{gather*}
                d_B(v) = X \text{, dove $X$ è l'insieme associato al nodo $u$ di $A$} \mid u R v, \\
                \text{con $u$ preso casualmente tra i nodi di $G_A$ bisimili a $v$.}
              \end{gather*}
              Per AFA esiste un'unica decorazione di $G_B$, quindi si deve avere, alternativamente, per ogni nodo $v$ di $G_B$:
              \begin{itemize}
                  \item $\exists ! u$ nodo di $G_A \mid u R v$;
                  \item $\forall u$ nodo di $G_A \mid u R v$, la decorazione di $G_A$ associa a tutti gli $u$ lo stesso insieme.
              \end{itemize}
              Dunque l'ambiguità è risolta.
    \end{enumerate}
    \vspace*{-0.75cm}
\end{proof2}
Tenendo conto di quanto affermato nella sezione \ref{sec:graphs_sets}, il Teorema \ref{theo:bisi_iff_eqsets} dimostra che la bisimulazione può sostituire la relazione di uguaglianza tra insiemi quando questi sono rappresentati con APG \cite{dovier}.\\

Dopo questa considerazone possiamo dare la seguente definizione:
\begin{definition}
    \label{def:bisi_contraction}
    Sia $R$ una bisimulazione su $G = (V,E)$, dove $G$ è un grafo diretto. Supponiamo che $R$ sia una relazione di equivalenza. Il grafo, definito a partire dal grafo iniziale, ottenuto con una \emph{contrazione rispetto alla bisimulazione} $R$, è il grafo diretto $G_R = (V_R, E_R)$ definito come in \cite{gentilini}:
    \begin{itemize}
        \item $V_R = \{[m]_R, m \in V\}$;
        \item $[m]_R E_R [n]_R \iff \exists b \in [m]_R, c \in [n]_R \mid b E c$.
    \end{itemize}
    Si chiama \emph{classe} del nodo $a$ rispetto alla bisimulazione $R$, indicata con la notazione $[a]_R$, il nodo di $V_R$ in cui viene inserito il nodo $a$.
\end{definition}
La Definizione \ref{def:bisi_contraction} è di fondamentale importanza per la seguente osservazione:
\begin{proposition}
    Sia $G$ un grafo diretto, e sia $G_R$ come nella Definizione \ref{def:bisi_contraction}, per una bisimulazione $R$ qualsiasi. Allora $G, G_R$ sono bisimili.
    \label{prop:bisi_cont_bisi}
\end{proposition}
\begin{proof2}
    Sia ``$\equiv$'' la relazione binaria su $V\times V_R$ definita come segue:
    \begin{gather*}
        m \equiv M \iff M = [m]_R
    \end{gather*}
    Vogliamo dimostrare che tale relazione è una bisimulazione sui grafi $G, G_R$.\\
    Supponiamo che $x \equiv X$, e che $x E y$ per qualche $y \in V$. Chiamiamo $Y \coloneqq [y]_R$. Allora, per la Definizione \ref{def:bisi_contraction}, si ha $X E_R Y$. Inoltre vale banalmente $y \equiv Y$.\\
    Per dimostrare la seconda condizione caratteristica della bisimulazione, supponiamo che $x \equiv X$, e che $X E_R Y$ per qualche $Y \in V_R$. Sempre per la Definizione \ref{def:bisi_contraction} deve esistere un $y \in Y \mid (y \equiv Y \land x E y)$.
\end{proof2}
La Proposizione \ref{prop:bisi_cont_bisi} ha una conseguenza ovvia, che risulta evidente dal Teorema \ref{theo:bisi_iff_eqsets}:
\begin{corollary}
    Sia $R$ una bisimulazione che sia anche una relazione di equivalenza. Allora l'APG $(G, v)$ e l'APG $(G_R, [v]_R)$ rappresentano lo stesso insieme.
\end{corollary}
Quindi risulta naturale sfruttare le proprietà della bisimulazione per minimizzare la rappresentazione di insiemi, considerando che è sufficiente una bisimulazione sulla rappresentazione iniziale per ottenere una rappresentazione equivalente. Definiamo una relazione d'ordine sulle rappresentazioni:
\begin{definition}
    La rappresentazione $(G_a, v_a)$ di un insieme è \emph{minore} della rappresentazione equivalente $(G_b, v_b)$ se $|Va| < |Vb|$.\\
    Una rappresentazione è \emph{minima} se non esiste una rappresentazione equivalente minore.
\end{definition}
\begin{observation}
    La \emph{contrazione per bisimulazione} è una rappresentazione minore (o eventualmente uguale) di quella iniziale.
\end{observation}
Concludiamo la sezione con il seguente risultato, che stabilisce in modo univoco la bisimulazione prescelta per minimizzare la rappresentazione di un dato insieme:
\begin{theorem}
    Sia $(G,v)$ un APG rappresentante un insieme. Sia $R_M$ la bisimulazione massima su $(G,v)$. Allora la contrazione per bisimulazione indotta da $R_M$ su $(G,v)$ fornisce la rappresentazione minima dell'insieme.
\end{theorem}
\begin{proof2}
    Supponiamo per assurdo che esista una bisimulazione $R_V$ su $(G,v)$ che fornisce una contrazione avente un numero di nodi strettamente inferiore alla contrazione indotta da $R_M$. Ma questo implica che esistono almeno due nodi di $G$ che sono in relazione secondo $R_V$ e non secondo $R_M$. Chiaramente questa deduzione è in contrasto con il fatto che $R_M$ è la bisimulazione massima.\\
    Supponiamo per assurdo che, dopo la contrazione indotta da $R_M$, sia possibile trovare una nuova bisimulazione $R_O$ su $(G_{R_M}, [v]_{R_M})$ che induca una contrazione avente un numero di nodi strettamente inferiore a quello di $(G_{R_M}, [v]_{R_M})$. Chiaramente $R_O \subset V{R_M} \times V{R_M}$.\\
    Definiamo una nuova bisimulazione $R_{\widetilde{M}} \subset V\times V$ tale che
    \begin{gather*}
        x R_{\widetilde{M}} y \iff (x R_M y \lor [x]_{R_M} R_O [y]_{R_M})
    \end{gather*}
    Per definizione di bisimulazione massima bisogna avere $R_{\widetilde{M}} \subset R_M$, quindi non è possibile che la contrazione indotta da $R_O$ sia una rappresentazione minore di quella indotta da $R_M$.
\end{proof2}
