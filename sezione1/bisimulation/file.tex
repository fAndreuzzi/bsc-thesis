\subsection{Bisimulazione}
In questa sezione introdurremo la definizione di \emph{bisimulazione} ed alcune proprietà. Proseguiremo con la definizione di \emph{massima bisimulazione}, ed infine proporremo alcune osservazioni che legano la bisimulazione e la teoria degli insiemi.

\subsubsection{Definizione e risultati preliminari}
\begin{definition}
    Siano $G_1 = (V_1,E_1), G_2 = (V_2,E_2)$ due grafi diretti. Una relazione binaria $\mathcal{R}: V_1 \times V_2$ è una \emph{bisimulazione} su $G_1, G_2$ se $\forall a \in V_1, b \in V_2$ valgono congiuntamente le seguenti proprietà:
    \begin{itemize}
        \item $a \mathcal{R} b, \langle a, a' \rangle_{E_1} \implies \exists b' \in V_2 \mid a' \mathcal{R} b' \land \langle b, b' \rangle_{E_2}$
        \item $a \mathcal{R} b, \langle b, b' \rangle_{E_2} \implies \exists a' \in V_1 \mid a' \mathcal{R} b' \land \langle a, a' \rangle_{E_1}$
    \end{itemize}
    Analogamente si definisce una bisimulazione su un unico grafo diretto $G$, ponendo $G_1 = G_2 = G$.
\end{definition}

Nei prossimi paragrafi osserveremo che l'esistenza di una bisimulazione tra due grafi è un'informazione rilevante se siamo interessati agli insiemi che rappresentano.
\begin{definition}
    Siano $G_1 = (V_1,E_1), G_2 = (V_2,E_2)$ due grafi. Essi sono \emph{bisimili} se esiste una bisimulazione su $G_1, G_2$.
    Due APG $(G_1, v_1), (G_2, v_2)$ sono \emph{bisimili} se $G_1, G_2$ sono bisimili e vale $v_1 \mathcal{R} v_2$ per almeno una bisimulazione $\mathcal{R}$ su $G_1, G_2$.
\end{definition}

\begin{observation}
    Una bisimulazione può non essere riflessiva, simmetrica, nè transitiva.
\end{observation}

\begin{example}
    La relazione $a R b \iff$ ``$a,b$ sono lo stesso nodo'' su un grafo qualsiasi è una bisimulazione riflessiva, simmetrica e transitiva.

    La relazione $R = \emptyset$ è una bisimulazione su un grafo qualsiasi, ma non è riflessiva.

    La relazione $R = \{(a,a),(b,b),(c,c),(d,d),(a,b),(b,c),(c,d)\}$ sul grafo $G = (\{a,b,c,d\}, \{(a,b),(b,c),(c,d),(d,d)\})$ è una bisimulazione, ed è solamente riflessiva.
\end{example}

Dalla definizione di bisimulazione possiamo dedurre una proprietà interessante delle chiusure:
\begin{theorem}
    Sia $\mathcal{R}$ una bisimulazione sul grafo diretto $G$. La sua chiusura riflessiva, simmetrica o transitiva è ancora una bisimulazione su $G$.
    \label{theo:bisi_chiusura}
\end{theorem}
\begin{proof2}
    Consideriamo separatamente le tre relazioni $\mathcal{R}_r, \mathcal{R}_s, \mathcal{R}_t$, rispettivamente la chiusura riflessiva, simmetrica e transitiva. Per ogni caso consideriamo solamente le coppie aggiunte dalla chiusura:
    \begin{itemize}
        \item $\mathcal{R}_r$: Per definizione $\mathcal{R} \subseteq \mathcal{R}_r$, quindi è sufficiente dimostrare che $\mathcal{R}_r$ è una bisimulazione quando gli argomenti $u,v \in V$ non sono distinti.

        Sia $u \in V$. Chiaramente per definizione di $\mathcal{R}_r$ si ha $u \mathcal{R}_r u$. Ma $\forall u' \in V \mid \langle u, u' \rangle$ vale (sempre per definizione di $\mathcal{R}_r$) $u' \mathcal{R}_r u'$.
        \item $\mathcal{R}_s$: Per definizione $\mathcal{R} \subset \mathcal{R}_s$, quindi è sufficiente dimostrare che $\mathcal{R}_s$ è una bisimulazione quando per $u,v \in V$ si ha $u \mathcal{R} v$ ma non $v \mathcal{R} u$.

        Sia $(u,v) \in V\times V$. Allora:
        \begin{center}
            $u \mathcal{R}_s v \implies u \mathcal{R} v \lor v \mathcal{R} u$
        \end{center}
        Supponiamo ad esempio che $v \mathcal{R} u$.
        \begin{align*}
            &\implies \forall v' \in V \mid \langle v, v' \rangle \,\,\exists u' \in V, \langle u, u' \rangle \land v' \mathcal{R} u'\\
            &\implies u' \mathcal{R}_s v'
        \end{align*}
        e
        \begin{align*}
            &\implies \forall u' \in V\mid \langle u, u' \rangle \,\,\exists v' \in V, \langle v, v' \rangle \land v' \mathcal{R} u'\\
            &\implies u' \mathcal{R}_s v'
        \end{align*}
        cioè sono dimostrate le due condizioni caratteristiche della bisimulazione.\\
        La dimostrazione è analoga se $u \mathcal{R} v$.
        \item $\mathcal{R}_t$: Per definizione $b \subset \mathcal{R}_t$, quindi è sufficiente dimostrare che $\mathcal{R}_t$ è una bisimulazione quando per gli argomenti $u,v,z \in V$ si ha $u \mathcal{R} v$, $v \mathcal{R} z$ ma non $u \mathcal{R} z$.\\
        Sia $(u,v,z) \in V^3$ una terna con questa proprietà. Allora:
        \begin{gather*}
            \forall u' \in V\mid \langle u, u' \rangle \,\, \exists v' \in V, \langle v, v' \rangle \land u' \mathcal{R} v'
        \end{gather*}
        Inoltre $\exists z' \mid \langle z, z' \rangle \land v' \mathcal{R} z'$.\\
        Riordinando si ha $u' \mathcal{R} v', v' \mathcal{R} z'$. Allora per definizione di $b_t, \, u' \mathcal{R}_t z'$.\\
        In modo speculare si ottiene la seconda condizione caratteristica della bisimulazione.
    \end{itemize}
    \vspace*{-0.75cm}
\end{proof2}

Da questa proposizione si deduce il seguente corollario, che risulta dall'applicazione iterata delle tre chiusure viste in precedenza:
\begin{corollary}
    Ad ogni bisimulazione $\mathcal{R}$ è possibile associare una bisimulazione $\widetilde{\mathcal{R}} \supseteq \mathcal{R}$ che è anche una relazione di equivalenza (Definizione \emph{\ref{def:eq_rel}}).
    \label{cor:bisimulation_eqrel}
\end{corollary}

Concludiamo la sezione relativa ai risultati generali sulla bisimulazione con la seguente proposizione, che sarà utile nel seguito:
\begin{proposition}
    Siano $\mathcal{R}_1, \mathcal{R}_2$ due bisimulazioni su $G_1, G_2$. Allora $\mathcal{R} = \mathcal{R}_1 \cup \mathcal{R}_2$ è ancora una bisimulazione.
    \label{obs:bisimulation_union}
\end{proposition}
\begin{proof2}
    Siano $u,v \mid u \mathcal{R} v$; allora deve valere $u \mathcal{R}_1 v \lor u \mathcal{R}_2 v$. Sia $u' \mid \langle u, u' \rangle$. Ma quindi $\exists v' \mid \langle v,v' \rangle \land u' \mathcal{R}_{1|2} v'$.
\end{proof2}

\subsubsection{Massima bisimulazione}
\label{sec:bisi_max}
Definiamo ora il concetto di \emph{massima bisimulazione}, che sarà l'argomento principale di questo elaborato:
\begin{definition}
    Una bisimulazione $\mathcal{R}_M$ su $G_1, G_2$ è la \emph{bisimulazione massima} su $G_1, G_2$ se per ogni bisimulazione $\mathcal{R}$ su $G_1,G_2 \,\, u \mathcal{R} v \implies u \mathcal{R}_M v$, o equivalentemente $\mathcal{R}_M \supseteq \mathcal{R}$ per ogni bisimulazione $\mathcal{R}$ su $G_1, G_2$.
\end{definition}

\begin{observation}
    Sia $\mathcal{R}_M$ la massima bisimulazione su un grafo diretto $G$. Allora per ogni bisimulazione $\mathcal{R}$ su $G$ si ha $|\mathcal{R}_M| \geq |\mathcal{R}|$.
\end{observation}

Naturalmente la massima bisimulazione dipende dai due grafi presi in esame. Deduciamo alcune caratteristiche immediate:
\begin{proposition}
    Valgono le seguenti proprietà:
    \begin{enumerate}
        \item La massima bisimulazione su due grafi $G_1,G_2$ è unica;
        \item La massima bisimulazione è una relazione di equivalenza.
    \end{enumerate}
    \vspace*{-0.3cm}
    \label{prop:bisi_max_equi}
\end{proposition}
\begin{proof2}
    Le proprietà seguono dal Corollario \ref{cor:bisimulation_eqrel} e dall'Osservazione \ref{obs:bisimulation_union}:
    \begin{enumerate}
        \item Supponiamo per assurdo che esistano due massime bisimulazioni $\mathcal{R}_{M_1}, \mathcal{R}_{M_2}$. La loro unione è ancora una bisimulazione per l'Osservazione \ref{obs:bisimulation_union}, ed ha cardinalità maggiore (se così non fosse le due bisimulazioni coinciderebbero).
        \item Se per assurdo la massima bisimulazione non fosse una relazione di equivalenza, potremmo considerare la sua chiusura riflessiva, simmetrica e transitiva, che avrebbe cardinalità maggiore o uguale alla supposta massima bisimulazione e sarebbe per costruzione una relazione di equivalenza.
    \end{enumerate}
    \vspace*{-0.7cm}
\end{proof2}
Naturalmente il concetto di \emph{massima bisimulazione} può essere definito anche su unico grafo diretto $G$. Questo caso si rivelerà di grande interesse nel seguito. Per ora dimostriamo il seguente risultato:
\begin{theorem}
    Sia $G=(V,E)$ un grafo diretto finito. Allora esiste la massima bisimulazione su $G$.
\end{theorem}
\begin{proof2}
    Può esistere solamente un numero finito di relazioni binarie su $G$, e questo numero fornisce un limite superiore al numero massimo di bisimulazioni su $G$.
    Allora possiamo considerare l'unione di questo numero finito di bisimulazioni, che sarà chiaramente la massima bisimulazione.

    Abbiamo almeno una bisimulazione non banale (ovvero diversa da $\emptyset$) che possiamo ricavare dalla chiusura riflessiva di $\emptyset$ (Teorema \ref{theo:bisi_chiusura}), ovvero la relazione $\mathcal{R} = \{(u,u) \mid u \in V\}$.
\end{proof2}

\subsubsection{Interpretazione insiemistica della bisimulazione}
Mostriamo ora una conseguenza diretta dell'Assioma di Estensionalità e di AFA (Sezione \ref{sec:graphs_sets}):
\begin{theorem}
    Due APG rappresentano lo stesso insieme se e solo se sono bisimili.
    \label{theo:bisi_iff_eqsets}
\end{theorem}
\begin{proof2}
    Siano $G_A = (V_A, E_A), G_B = (V_B, E_B)$. Dimostriamo separatamente le due implicazioni:
    \begin{enumerate}
        \item[$(\implies)$] Osserviamo innanzitutto che la relazione binaria $\equiv \,: V_A \times V_B$ definita come:
        \begin{gather*}
            a \equiv b \iff ``\text{le decorazioni di } G_A,G_B \text{ associano ad } a,b \text{ lo stesso insieme}''
        \end{gather*}
        è una bisimulazione sui grafi $G_A, G_B$.\\
        Chiaramente se $a \equiv b$ e $\langle a, a' \rangle$ si ha:
        \begin{itemize}
            \item $a' \in a$, associando ad $a,a'$ gli insiemi che rappresentano secondo la decorazione (unica per AFA);
            \item $a,b$ rappresentano lo stesso insieme.
        \end{itemize}
        Quindi per l'Assioma di Estensionalità $\exists b' \in b\mid b',a'$ rappresentano lo stesso insieme, cioè $\langle b, b' \rangle$ e $a' \equiv b'$. Si procede specularmente per la seconda condizione caratteristica della bisimulazione.\\
        La relazione ``$\equiv$'' è quindi una bisimulazione sugli APG $G_A,G_B$ quando si assume per ipotesi che rappresentino lo stesso insieme.
        \item[$(\impliedby)$] Sia $\mathcal{R}$ una bisimulazione su $G_A,G_B$. Consideriamo la decorazione $d_A$ (l'unica) di $G_A$. Vogliamo definire una decorazione per $G_B$, e dimostrare che i due grafi con le rispettive decorazioni sono due immagini dello stesso insieme. Dalla possibilità di operare questo procedimento, dall'Assioma di Estensionalità e da AFA potremo dedurre l'uguaglianza degli insiemi rappresentati.

        Definiamo la decorazione $d_B$ come segue:
        \begin{gather*}
            d_B(v) = d_A(u), \text{con $u$ nodo di $G_A$} \mid u \mathcal{R} v
        \end{gather*}
        \begin{observation*}
            Per ogni nodo $v$ di $B$ deve esistere almeno un nodo $u$ di $G_A \mid u \mathcal{R} v$ perchè si suppone che i due APG siano bisimili.
        \end{observation*}
        Dimostriamo che $d_B$ è una decorazione di $G_B$. In altre parole vogliamo dimostrare che per ogni nodo $v$ di $G_B$ l'insieme $d_B(v)$ contiene tutti e soli gli insiemi $d_B(v')$ dove $v'$ è figlio di $v$.
        \begin{itemize}
            \item Supponiamo per assurdo che tra i figli di $v$ ``manchi" il nodo corrispondente ad un elemento $X \in d_B(v)$. Poichè la decorazione di $G_A$ è ben definita, il nodo $u$ di $G_A \mid u \mathcal{R} v$ deve avere un figlio corrispondente a $X$, che chiameremo $u'$, per cui vale quindi $d_A(u') = X$. Ma $u \mathcal{R} b \land \langle u, u' \rangle \implies \exists v' \mid \langle v, v' \rangle, u' \mathcal{R} v'$. Cioè $d_B(v') = d_A(u') = X$. Dunque il nodo mancante è stato trovato.
            \item Supponiamo per assurdo che tra i figli di $v$ ci sia un nodo ``in più'', ovvero un nodo $v' \mid \langle v, v' \rangle$ e $d_B(v') = Y$ con $Y \notin d(v)$. Ma allora, considerando un nodo $u$ di $G_A \mid u \mathcal{R} v$, dovrebbe esistere un $u' \mid \langle u, u' \rangle$ e $u' \mathcal{R} v'$, cioè $d_A(u') = d_B(v') = Y$.
            Ma allora $Y \in d_A(u) = d_B(v)$, deduzione che è chiaramente in contrasto con l'ipotesi.
        \end{itemize}
        \accente possibile che ci siano due nodi di $a_1, a_2$ di $G_A$ ed un nodo $b$ di $G_B$ per cui vale $a_1 \mathcal{R} b \land a_2 \mathcal{R} b$. In questo caso la decorazione definita è ambigua. Per questo motivo correggiamo la formulazione di $d_B$ come segue:
        \begin{gather*}
        d_B(v) = X \text{, dove $X$ è l'insieme associato al nodo $u$ di $A$} \mid u \mathcal{R} v, \\
        \text{con $u$ preso casualmente tra i nodi di $G_A$ bisimili a $v$.}
        \end{gather*}
        Per AFA esiste un'unica decorazione di $G_B$, quindi si deve avere, alternativamente, per ogni nodo $v$ di $G_B$:
        \begin{itemize}
            \item $\exists ! u \in V_A \mid u \mathcal{R} v$;
            \item $\forall u \in V_A \mid u \mathcal{R} v$, la decorazione di $G_A$ associa a tutti gli $u$ lo stesso insieme.
        \end{itemize}
        Dunque l'ambiguità è risolta.
    \end{enumerate}
    \vspace*{-0.75cm}
\end{proof2}
Tenendo conto di quanto affermato nella sezione \ref{sec:graphs_sets}, il Teorema \ref{theo:bisi_iff_eqsets} dimostra che la bisimulazione può sostituire la relazione di uguaglianza tra insiemi quando questi sono rappresentati con APG \cite{dovier}.

Dopo questa considerazone possiamo dare la seguente definizione:
\begin{definition}
    \label{def:bisi_contraction}
    Sia $\mathcal{R}$ una bisimulazione su $G = (V,E)$, dove $G$ è un grafo diretto. Supponiamo che $\mathcal{R}$ sia una relazione di equivalenza. Il grafo, definito a partire dal grafo iniziale, ottenuto con una \emph{contrazione rispetto alla bisimulazione} $\mathcal{R}$ \cite{gentilini}, è il grafo diretto $G_{\mathcal{R}} = (V_{\mathcal{R}}, E_{\mathcal{R}})$:
    \begin{itemize}
        \item $V_{\mathcal{R}} = \{[m]_{\mathcal{R}}, m \in V\}$;
        \item $\langle [m]_{\mathcal{R}}, [n]_{\mathcal{R}} \rangle \in E_{\mathcal{R}} \iff \exists b \in [m]_{\mathcal{R}}, c \in [n]_{\mathcal{R}} \mid \langle b, c \rangle \in E$.
    \end{itemize}
    Si chiama \emph{classe} del nodo $a$ rispetto alla bisimulazione $\mathcal{R}$, indicata con la notazione $[a]_{\mathcal{R}}$, il nodo di $V_{\mathcal{R}}$ in cui viene inserito il nodo $a$.
\end{definition}

La Definizione \ref{def:bisi_contraction} è di fondamentale importanza per la seguente osservazione:
\begin{proposition}
    Sia $G$ un grafo diretto, e sia $G_{\mathcal{R}}$ come nella Definizione \ref{def:bisi_contraction}, per una bisimulazione $R$ qualsiasi. Allora $G, G_{\mathcal{R}}$ sono bisimili.
    \label{prop:bisi_cont_bisi}
\end{proposition}
\begin{proof2}
    Sia ``$\equiv$'' la relazione binaria su $V\times V_{\mathcal{R}}$ definita come segue:
    \begin{gather*}
        m \equiv M \iff M = [m]_{\mathcal{R}}
    \end{gather*}
    Vogliamo dimostrare che tale relazione è una bisimulazione sui grafi $G, G_{\mathcal{R}}$. Supponiamo che $x \equiv X$, e che $\langle x,y \rangle \in E$ per qualche $y \in V$. Chiamiamo $Y \coloneqq [y]_{\mathcal{R}}$. Allora, per la Definizione \ref{def:bisi_contraction}, si ha $\langle X, Y \rangle \in E_{\mathcal{R}}$. Inoltre vale banalmente $y \equiv Y$.

    Per dimostrare la seconda condizione caratteristica della bisimulazione, supponiamo che $x \equiv X$, e che $\langle X, Y \rangle \in E_{\mathcal{R}}$ per qualche $Y \in V_{\mathcal{R}}$. Sempre per la Definizione \ref{def:bisi_contraction} deve esistere un $y \in Y \mid (y \equiv Y \land \langle x, y \rangle \in E)$.
\end{proof2}

La Proposizione \ref{prop:bisi_cont_bisi} ha una conseguenza ovvia, che risulta evidente dal Teorema \ref{theo:bisi_iff_eqsets}:
\begin{corollary}
    Sia $R$ una bisimulazione che sia anche una relazione di equivalenza. Allora l'APG $(G, v)$ e l'APG $(G_R, [v]_R)$ rappresentano lo stesso insieme.
\end{corollary}

Possiamo quindi sfruttare le proprietà della bisimulazione per rappresentare insiemi in una forma minimizzata: è sufficiente ricavare una bisimulazione (che sia anche una relazione di equivalenza) sul grafo che rappresenta l'insieme.

Poichè in generale un grafo ammette più di una bisimulazione (che sia anche una relazione di equivalenza), definiamo una relazione d'ordine sulle rappresentazioni:
\begin{definition}
    La rappresentazione $(G_a, v_a)$ di un insieme è \emph{minore} della rappresentazione equivalente $(G_b, v_b)$ se $|Va| < |Vb|$.

    Una rappresentazione è \emph{minima} se non esiste una rappresentazione equivalente minore.
\end{definition}

\begin{observation}
    La \emph{contrazione per bisimulazione} è una rappresentazione minore (o eventualmente uguale) di quella iniziale.
\end{observation}

Concludiamo la sezione con il seguente risultato, che stabilisce in modo univoco la bisimulazione prescelta per minimizzare la rappresentazione di un dato insieme:
\begin{theorem}
    Sia $(G,v)$ un APG rappresentante un insieme. Sia $\mathcal{R}_M$ la massima bisimulazione su $(G,v)$. Allora la contrazione per bisimulazione indotta da $\mathcal{R}_M$ su $(G,v)$ fornisce la rappresentazione minima dell'insieme.
\end{theorem}
\begin{proof2}
    Supponiamo per assurdo che esista una bisimulazione $\mathcal{R}_V$ su $(G,v)$ che fornisce una contrazione avente un numero di nodi strettamente inferiore alla contrazione indotta da $\mathcal{R}_M$. Ma questo implica che esistono almeno due nodi di $G$ che sono in relazione secondo $\mathcal{R}_V$ e non secondo $\mathcal{R}_M$. Chiaramente questa deduzione è in contrasto con il fatto che $\mathcal{R}_M$ è la massima bisimulazione.

    Supponiamo per assurdo che, dopo la contrazione indotta da $\mathcal{R}_M$, sia possibile trovare una nuova bisimulazione $\mathcal{R}_O$ su $(G_{\mathcal{R}_M}, [v]_{\mathcal{R}_M})$ che induca una contrazione avente un numero di nodi strettamente inferiore a quello di $(G_{\mathcal{R}_M}, [v]_{\mathcal{R}_M})$. Chiaramente $\mathcal{R}_O \subset V_{\mathcal{R}_M} \times V_{\mathcal{R}_M}$.

    Definiamo una nuova bisimulazione $\mathcal{R}_{\widetilde{M}} \subset V\times V$ tale che
    \begin{gather*}
        x \mathcal{R}_{\widetilde{M}} y \iff (x \mathcal{R}_M y \lor [x]_{\mathcal{R}_M} \mathcal{R}_O [y]_{\mathcal{R}_M})
    \end{gather*}
    Per definizione di massima bisimulazione è necessario che valga $\mathcal{R}_{\widetilde{M}} \subset \mathcal{R}_M$, quindi non è possibile che la contrazione indotta da $\mathcal{R}_O$ sia una rappresentazione minore di quella indotta da $\mathcal{R}_M$.
\end{proof2}
