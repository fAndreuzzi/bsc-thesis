\subsection{Equivalenza tra RSCP e massima bisimulazione}
Dimostriamo innanzitutto il seguente risultato preliminare presentato in \cite{gentilini}:
\begin{proposition}
    Sia $G = (V,E)$. Sia $X$ una partizione di $V$ stabile rispetto a $E$. Allora la relazione binaria $R$ su $V$ definita come:
    \begin{gather*}
        a R b \iff [a]_X = [b]_X
    \end{gather*}
    è una bisimulazione su $G$.
    \label{prop:part_induce_bisi}
\end{proposition}
\begin{proof2}
    Siano $a,b \in G \mid a R b$, e sia $a' \mid a E a'$. Poichè $X$ è stabile, si ha che $[a]_X \subseteq E^{-1}([a']_X)$. Quindi $\exists b' \in [a']_X \mid b E b'$.\\
    L'altra condizione caratteristica della bisimulazione si dimostra in modo speculare.
\end{proof2}
In altre parole, una qualsiasi partizione stabile rispetto a $E$ induce su un grafo diretto una bisimulazione che può essere ricavata in modo banale.\\
Dimostriamo ora il risultato opposto:
\begin{proposition}
    Sia $R$ una bisimulazione su $G$ che sia anche una relazione di equivalenza. Allora la partizione i cui blocchi sono le classi di equivalenza di $R$ è stabile rispetto a $E$.
    \label{prop:bisi_induce_part}
\end{proposition}
\begin{proof2}
    Se per assurdo $X$ non fosse stabile esisterebbero due blocchi $x_1, x_2 \mid x_1 \,\,\cap E^{-1}(x_2)$ non è ne $x_1$ nè $\emptyset$. Chiamiamo $A$ questo insieme.\\
    Gli elementi $a$ di $A$ sono i nodi in $x_1 \mid \nexists b \in x_2$ per cui valga $a E b$. Ma poichè questi $a$ e gli $x \in x_1 - A$ si trovano all'interno dello stesso blocco $x_1$ deve valere $x R a$.\\
    Sia $x \in x_1 - A$, ed $y \in x_2 \mid x E y$. Poichè $\forall a \in A$ vale $x R a$, allora $\exists a' \mid a E a', a' R y$, cioè $[a']_X = [y]_X = x_2$. Quindi $A$ deve necessariamente essere vuoto.
\end{proof2}
Cioè una bisimulazione induce un partizionamento stabile rispetto a $E$ dei nodi del grafo. Vale il seguente corollario:
\begin{corollary}
    Determinare la massima bisimulazione su un grafo diretto $G = (V,E)$ e trovare $\rscp(E)$ di $V$ sono problemi equivalenti.
\end{corollary}
\begin{proof2}
    Dimostriamo separatamente che la bisimulazione ricavata dalla $\rscp(E)$ è massima, e che la partizione ricavata dalla massima bisimulazione è la $\rscp(E)$.
    \begin{itemize}
        \item Sia $R_M$ la massima bisimulazione su $G$. Per la Proposizione \ref{prop:bisi_max_equi} è una relazione di equivalenza. Per la Proposizione \ref{prop:bisi_induce_part} è possibile determinare una partizione $X$ stabile rispetto a $E$.\\
              Supponiamo per assurdo che $X$ non sia $\rscp(E)$ di $V$, quindi esiste una partizione $\widetilde{X}$ stabile rispetto a $E$ che ha meno blocchi di quanti ne ha $X$. Ma per la Proposizione \ref{prop:part_induce_bisi} da
              $\widetilde{X}$ è possibile ricavare una bisimulazione $\widetilde{R}$ su $G$. Ma quindi $|\widetilde{R}| > |R|$, che è assurdo.
        \item Sia $X$ la $\rscp(E)$ di V. Supponiamo per assurdo che la bisimulazione $R$ ricavata da $X$ come nella Proposizione \ref{prop:part_induce_bisi} non sia massima. Allora deve esistere un'altra bisimulazione $\widetilde{R}$ che
              sia massima. Ma da questa si può ricavare, come nella Proposizione \ref{prop:bisi_induce_part}, una partizione $\widetilde{X}$ stabile rispetto a $E$ per cui vale $|\widetilde{X}| \leq |X|$. Ma questo è assurdo.
    \end{itemize}
    \vspace*{-0.75cm}
\end{proof2}
