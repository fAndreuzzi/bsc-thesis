\subsection{Algoritmo di Dovier-Piazza-Policriti}
\label{sec:dovier_piazza_policriti}
In questa sezione presenteremo l'algoritmo di Dovier-Piazza-Policriti \cite{dovier}, più recente rispetto all'algoritmo di Paige-Tarjan. La differenza fondamentale è l'introduzione di un ordinamento parziale tra i nodi del grafo, il \emph{rango}, che quantifica sostanzialmente la distanza di un nodo dal \emph{pozzo} più lontano raggiungibile percorrendo gli archi (si ricordi che un \emph{pozzo} è un nodo da cui non escono archi). Utilizzando questo ordinamento è possibile rifinire la partizione in modo più oculato, arrivando al risultato con meno chiamate a \splitfunc. La complessità computazionale resta la stessa dell'algoritmo di Paige-Tarjan, ma come vedremo nei risultati sperimentali in molti casi l'algoritmo di Dovier-Piazza-Policriti risulta più economico del suo predecessore.

La sezione è organizzata come segue: innanzitutto daremo alcune nozioni fondamentali per la comprensione dell'algoritmo, e alcuni risultati che risulteranno utili nel seguito; dopodichè esamineremo lo pseudocodice, ed analizzeremo la complessità e la correttezza dell'algoritmo.

\subsubsection{Nozioni preliminari}
\label{sec:fba_preliminari}
Consideriamo la seguente definizione, che ci consentirà di esprimere in modo più compatto alcune espressioni nel seguito:
\begin{definition}
    \label{def:grafo_restr}
    Sia $G = (V,E)$ un grafo diretto. Sia $n \in V$. Il grafo $\restr{G}{n}$, letto ``\emph{il sottografo di $G$ raggiungibile da $n$}'', è definito come segue:
    \begin{gather*}
        \restr{G}{n} = \left(\restr{N}{n}, E \cap \left(\restr{N}{n} \times \restr{N}{n}\right)\right)
    \end{gather*}
    dove $\restr{N}{n}$ è il sottoinsieme di $V$ dei nodi raggiungibili da $n$.
\end{definition}

Procediamo con un concetto fondamentale per la definizione del \emph{rango}, l'ordinamento parziale sull'insieme dei nodi menzionato nell'introduzione della sezione:
\begin{definition}
    La parte \emph{ben fondata} (\emph{well founded} in inglese) di un grafo diretto $G = (V,E)$ è:
    \begin{gather*}
        \texttt{WF}(G) = \left\{n \in V \mid \restr{G}{n} \text{ è aciclico}\right\}.
    \end{gather*}
    La parte \emph{non ben fondata} è definita chiaramente come:
    \begin{gather*}
        \texttt{NWF}(G) = V - \texttt{WF}(G).
    \end{gather*}
\end{definition}

Vale il seguente risultato immediato (si ricordi che con la notazione $[v], v \in V$ ci riferiamo alla SCC di $V$ cui appartiene il nodo $v$):
\begin{observation}
    Condizione necessaria affinchè un vertice $n$ possa appartenere a $WF(G)$ è che $|[n]| = 1$.
\end{observation}
\begin{proof2}
    Supponiamo che $n \in \texttt{WF}(G)$. Questo vuol dire che da tutti i nodi raggiungibili da $n$ non si può ritornare in $n$ (perchè non si hanno cicli). Ma allora non ci può essere un altro nodo nella SCC di $n$.
\end{proof2}

Introduciamo la funzione \rankfunc, che ricopre un ruolo primario nell'Algoritmo BFA:
\begin{definition}
    \label{def:rank}
    Sia $G = (V,E)$ un grafo diretto. La funzione $\rankfunc: V \to \mathbb{N} \cup \{0, -\infty\}$ è definita come segue:
    \begin{gather*}
        \rankfunc(n) = \begin{cases}
            0 &\text{se $n$ è un pozzo di $G$}\\
            -\infty &\parbox[t]{.32\textwidth}{se $[n]$ è un pozzo di $G^{\scalebox{.5}{SCC}}$,
                        e $n$ non è un pozzo di $G$}\\
            \max(\{1 + \rankfunc(m) \mid m \in \mathcal{N_{\scalebox{0.5}{WF}}}(n)\}\\
            \,\,\,\,\,\,\,\,\,\,\,\,\,\, \cup \,\, \{\rankfunc(m) \mid m \in \mathcal{N_{\scalebox{0.5}{NWF}}}(n)\}) &\text{altrimenti}
        \end{cases}
    \end{gather*}
    Abbiamo utilizzato per comodità le funzioni $\mathcal{N}_{\scalebox{0.5}{WF}}(n), \mathcal{N}_{\scalebox{0.5}{NWF}}(n)$ definite come segue:
    \begin{align*}
        &\mathcal{N}_{\scalebox{0.5}{WF}}(n) = \{m \in \texttt{WF}(G) \mid [n] \sccto [m]\}\\
        &\mathcal{N}_{\scalebox{0.5}{NWF}}(n) = \{m \in V - \texttt{WF}(G) \mid [n] \sccto [m]\}
    \end{align*}
    Esse associano ad un nodo $n \in V$ l'insieme delle SCC raggiungibili da $[n]$, rispettivamente contenute in $\texttt{WF}(G), V - \texttt{WF}(G)$.
\end{definition}
\begin{observation}
    Sia $n \in V$, e supponiamo $\rankfunc(n) = k$. Allora:
    \begin{gather*}
        \rankfunc(m) = k \,\,\forall m \in [n]
    \end{gather*}
\end{observation}
\begin{proof2}
    Se $k = 0$ allora $n$ è una pozzo, per cui $[n]$ ha cardinalità 1, quindi la tesi vale banalmente. Se $k = -\infty$, nessuno dei nodi in $[n]$ può essere un pozzo di $G$, per cui hanno tutti rango $-\infty$. Infine se $k > 0$ basta osservare che il rango massimo si ``propaga'' tra nodi della stessa SCC, com'è evidente dalla Definizione \ref{def:rank}.
\end{proof2}

Se un grafo è \emph{ben fondato}, o se restringiamo il dominio ad un sottografo \emph{ben fondato}, possiamo dare una formulazione alternativa della funzione \rankfunc:
\begin{gather*}
        \rankfunc(n) = \begin{cases}
            0 &\text{se $n$ è un pozzo di $G$}\\
            1 + \max\{\rankfunc(m) \mid n E m\} &\text{altrimenti}
        \end{cases}
\end{gather*}
Questa forma sarà utile per semplificare alcune delle dimostrazioni che seguono.

Riportiamo e dimostriamo alcuni risultati \cite{dovier} che renderanno più agile l'analisi successiva. Il motivo per cui troviamo interessante introdurre il rango nella trattazione algoritmica della massima bisimulazione è sintetizzato dal seguente risultato:
\begin{theorem}
    \label{theo:bisi_rank}
    Sia $G = (V,E)$. Sia ``\,$\equiv$'' la massima bisimulazione su $G$. Siano $m,n \in V$. Allora
    \begin{gather*}
        m \equiv n \implies \rankfunc(m) = \rankfunc(n)
    \end{gather*}
\end{theorem}
La dimostrazione del Teorema \ref{theo:bisi_rank} è piuttosto lunga, per cui abbiamo preferito riportarla a pezzi. Cominciamo con la prima parte:
\begin{proposition} \label{prop:rank_bisi_imp_wf}
    Sia $G = (V,E)$. Sia ``\,$\equiv$'' la massima bisimulazione su $G$. Siano $m,n \in \texttt{WF}(G)$. Allora $m \equiv n \implies \rankfunc(m) = \rankfunc(n)$.
\end{proposition}
\begin{proof2}
    Procediamo per induzione su $\rankfunc(m)$. Si ricordi che per il Teorema \ref{theo:bisi_iff_eqsets} gli APG aventi origine in $m,n$ rappresentano lo stesso insieme. Se $\rankfunc(m) = 0, m$ rappresenta l'insieme vuoto. Ma allora lo stesso vale per $n$, per cui anche $\rankfunc(n) = 0$.

    Supponiamo (ipotesi induttiva) che due nodi bisimili di rango minore o uguale a $k-1$ abbiano sempre rango uguale. Sia $\rankfunc(m) = k$, e sia $m' \in V$ tale che $\langle m, m'\rangle \in E, \rankfunc(m') = k-1$. Allora deve esistere un nodo $n'$ tale che $\langle n, n'\rangle \in E, m' \equiv n'$, e quindi $\rankfunc(m') = \rankfunc(n')$. Ma allora $\rankfunc(n) \geq k-1 + 1 = k$, per la formulazione alternativa di \rankfunc. Analogamente si dimostra che $\rankfunc(m) \geq k$, per cui si ha la tesi.
\end{proof2}

La seconda parte del Teorema \ref{theo:bisi_rank} richiede un certo numero di sotto-risultati preliminari:
\begin{proposition}
    \label{prop:omega_rank}
    Sia $G = (V,E)$. Sia $m \in V$. Allora $\rankfunc(m) = -\infty$ se e solo se l'APG $\restr{G}{m}$ rappresenta l'insieme $\Omega$.
\end{proposition}
\begin{proof2}
    Supponiamo che $\rankfunc(m) = - \infty$. Si ricordi la caratterizzazione dell'insieme $\Omega$ (cioè l'insieme che contiene solamente se stesso \cite{aczel}): un APG rappresenta $\Omega$ se e solo se ogni nodo ha almeno un arco uscente. Dalle ipotesi fatte si deduce che $[m]$ non contiene solamente $m$ (altrimenti $m$ sarebbe un pozzo di $G$). Possiamo dimostrare che ogni nodo in $[m]$ deve avere almeno un arco uscente: se così non fosse il nodo non avrebbe modo di tornare all'interno della SCC $[m]$, per cui non potrebbe stare nella stessa SCC di $m$.

    Supponiamo ora che l'APG $\restr{G}{m}$ rappresenti l'insieme $\Omega$. Chiaramente $m$ non può essere un pozzo di $G$, altrimenti non rispetterebbe la caratterizzazione. Supponiamo per assurdo che $[m]$ non sia un pozzo di $G^{\superscc}$, cioè da un nodo di $[m]$ parte un arco verso un nodo $x \not\in [m]$. Si presentano quattro possibilità:
    \begin{enumerate}
        \item Da $x$ non esce alcun nodo: impossibile per la caratterizzazione di $\Omega$;
        \item Da $x$ esce un unico nodo che ha per destinazione $x$ stesso: $[x]$ contiene solamente $x$, ed è un pozzo di $G^{\superscc}$, per cui ha rango $-\infty$;
        \item Da $x$ esce un nodo che ha per destinazione un nodo di $[m]$: impossibile perchè si è supposto $x \not\in [m]$;
        \item Da $x$ esce un nodo che ha per destinazione un nodo $x' \not\in [m]$.
    \end{enumerate}
    Per cui è evidente che le uniche opzioni valide sono la 2. \hspace{-0.4cm} e la 4. Possiamo dimostrare che il blocco di $G^{\superscc}$ contenente una sequenza di nodi connessi secondo queste due regole è necessariamente un pozzo di $G^{\superscc}$: è possibile aggiungere un numero arbitrario di nodi che le rispettino, ma l'ultimo nodo dovrà avere un \emph{self loop}, oppure tornare in un nodo precedente della sequenza. In entrambi i casi i nodi coinvolti ottengono rango $-\infty$, per cui a cascata (per il terzo caso nella definizione di \rankfunc) i nodi precedenti della sequenza hanno tutti rango $-\infty$.
\end{proof2}

La seguente osservazione è una riformulazione di una deduzione proposta nella Sezione \ref{sec:base}:
\begin{observation}
    Sia $G = (V,E)$. Siano $m,n \in V$. Allora:
    \begin{gather*}
        \rankfunc(m) = \rankfunc(n) = 0 \implies m \equiv n
    \end{gather*}
\end{observation}
\begin{proof2}
    $m,n$ sono pozzi di $G$, per cui rappresentano l'insieme $\emptyset$. Allora $m,n$ sono bisimili per il Teorema \ref{theo:bisi_iff_eqsets}.
\end{proof2}

Possiamo dunque enunciare e dimostrare la parte mancante della dimostrazione:
\begin{proposition}
    \label{prop:bisi_rank_nwf}
    Con le stesse ipotesi della Proposizione \ref{prop:rank_bisi_imp_wf}, supponiamo che $m,n \in V - \texttt{WF}(G)$. Allora $m \equiv n \implies \rankfunc(m) = \rankfunc(n)$.
\end{proposition}
\begin{proof2}
    Sempre per il Teorema \ref{theo:bisi_iff_eqsets}, $m,n$ rappresentano lo stesso insieme. Se $\rankfunc(m) = -\infty$, per la Proposizione \ref{prop:omega_rank} $m$ rappresenta $\Omega$, per cui anche $n$ rappresenta $\Omega$. Ma allora $\rankfunc(n) = -\infty$.

    Se $\rankfunc(m) = h > 0$, per come è stato definita la funzione \rankfunc deve esistere un nodo \emph{ben fondato} $m'$ raggiungibile da $m$, non necessariamente in modo diretto, tale che $\rankfunc(m') = h-1$ (il rango aumenta solo in corrispondenza di archi verso nodi \emph{ben fondati}). Poichè $m \equiv n$ deve esistere un nodo \emph{ben fondato} $n'$ raggiungibile da $n$ tale che $m' \equiv n'$. Ma allora ($m',n' \in \texttt{WF}(G)$) $\rankfunc(n') = \rankfunc(m') = h-1$, e quindi $\rankfunc(n) \geq \rankfunc(m) = h$. Analogamente si dimostra che $\rankfunc(m) \geq \rankfunc(n)$, per cui si ha la tesi.
\end{proof2}

\begin{proof2}[Teorema \ref{theo:bisi_rank}\,]
    Due nodi bisimili devono essere entrambi \emph{ben fondati} oppure entrambi \emph{non ben fondati}. La tesi segue dunque dalle Proposizioni \ref{prop:rank_bisi_imp_wf}, \ref{prop:bisi_rank_nwf}.
\end{proof2}

Dal Teorema \ref{theo:bisi_rank} deduciamo che due nodi aventi rango differente sono sicuramente \textbf{non} bisimili. Inoltre nodi a basso rango non sono influenzati da quanto avviene ai nodi ad alto rango; per cui, invece di procedere in modo randomico, possiamo utilizzare la funzione \splitfunc sui blocchi in ordine crescente di rango. Ci aspettiamo con molta probabilità un risparmio di qualche tipo in termini di tempo di esecuzione rispetto all'algoritmo di Paige-Tarjan, in particolare su grafi che mostrano una struttura di qualche tipo; grafi generati in modo randomico, invece, hanno più probabilità di mostrare una varianza del rango poco significativa, per cui è possibile che l'algoritmo di Dovier-Piazza-Policriti impieghi più tempo.

Riportiamo un algoritmo per il calcolo del rango dei nodi di un grafo diretto \cite{dovier}. Ne analizziamo brevemente le caratteristiche discutendo alcuni semplici risultati:
\begin{algorithm}[t!]
    \caption{Compute-Rank}
    \label{alg:rank}
    \KwData{$G = (V,E)$}
    \SetKwProg{Fn}{function}{:}{end}
    \Fn{\textup{dfs-rank-visit}($G = (V,E), n$)}{
        n.color = GRAY\;
        \If{$|E(n)| == 0$}{
            n.rank = 0\;
        } \Else{
            max-rank = $-\infty$\;
            \tcp*[h]{Visitiamo $E(n)$ in ordine decrescente di \emph{ft}.}\\
            \ForEach{$v \in E(n)$}{
                \If{v.color == WHITE or v.color == GRAY or !v.wf}{
                    n.wf = false\;
                }
                \If{v.color == WHITE}{
                    dfs-rank-visit($G, v$)\;
                }
                \If{v.rank != None}{ \label{alg:set_rank}
                    \If{v.wf}{
                        n.rank = max(n.rank, v.rank + 1)\;
                    } \Else{
                        n.rank = max(n.rank, v.rank)\;
                    }
                }
            }
        }
        n.color = BLACK\;
    }
    \Begin{
        DFS($G^{-1}$)\;\label{alg:rank_inverse_dfs}
        \tcp*[h]{Resettiamo i colori prima della seconda DFS.}\\
        \ForEach{$n \in V$}{
            n.color = WHITE\;
        }
        \tcp*[h]{Visitiamo $V$ in ordine decrescente di \emph{ft}.}\\
        \ForEach{$n \in V$}{
            \If{n.color == WHITE}{
                dfs-rank-visit($G,n$)\;
            }
        }
    }
\end{algorithm}

\begin{observation}
    L'Algoritmo \ref{alg:rank} termina, essendo una DFS. Inoltre, per lo stesso motivo, l'algoritmo ha complessità $O(|V| + |E|)$.
\end{observation}
Nell'analisi dell'Algoritmo \ref{alg:rank} indicheremo con ``$v$.rank'' il rango del nodo $v$ impostato dall'algoritmo, e con ``$\rankfunc(v)$'' il rango corretto. Inoltre useremo la notazione ``$v$.\emph{ft}'' per indicare il \emph{finishing-time} del nodo $v$ relativo alla DFS su $G^{-1}$ (Riga \ref{alg:rank_inverse_dfs}).
\begin{theorem}
    \label{theo:rank_correct}
    Dopo l'esecuzione dell'Algoritmo \ref{alg:rank} il rango di ogni nodo di $G$ è corretto, cioè $\forall v \in V$ vale $\rankfunc(v) = v$.\emph{rank}.
\end{theorem}

Proponiamo innanzitutto alcuni risultati preliminari che consentono di semplificare la trattazione:
\begin{observation}
    \label{obs:rank_img_scc}
    Se $\langle v,u\rangle \in E$ e $v$.\emph{ft} $> u$.\emph{ft}, allora $u \in [v]$.
\end{observation}
\begin{proof2}
    Chiaramente da $v$ è possibile raggiungere $u$ per ipotesi. Consideriamo la DFS su $G^{-1}$: ovviamente si avrà $\langle u, v\rangle \in E^{-1}$. Ma se $v$ fosse stato raggiunto per la prima volta durante la visita in profondità di $u$, avremmo $u$.\emph{ft} $> v$.\emph{ft}. Allora $v$ viene raggiunto \emph{prima} di $u$, e l'unica possibilità per cui si possa avere $v$.ft $> u$.ft è che esista un percorso da $v$ a $u$ (in $G^{-1}$). Procedendo al contrario lungo questo cammino si verifica che da $u$ è possibile raggiungere $v$.
\end{proof2}

L'osservazione può essere riformulata in modo più esplitito:
\begin{corollary}
    Se durante la visita dell'immagine $v$ si incontra un vertice $u$ di colore bianco, $v$ è raggiungibile da $u$ e viceversa.
\end{corollary}

E generalizzata:
\begin{corollary}
    \label{cor:scc_minore_inglobamento}
    Se $v$.ft $> u$.ft, e $u$ è raggiungibile da $v$, allora $u \in [v]$.
\end{corollary}

Dall'Osservazione \ref{obs:rank_img_scc} possiamo dedurre ancora alcuni risultati immediati:
\begin{corollary}
    Se $\langle v,u\rangle \in E$ e $v$.ft $> u$.ft, allora $u,v \in \texttt{NWF}(G)$.
\end{corollary}

Si osservi che da quest'ultimo corollario si deduce che il criterio usato dall'Algoritmo \ref{alg:rank} per impostare il campo ``\emph{wf}'' dei nodi è corretto.
\begin{corollary}
    \label{cor:no_buchi_scc}
    Il \emph{finishing-time} dei nodi di una SCC è un intervallo di interi senza buchi.
\end{corollary}
\begin{proof2}
    Supponiamo per assurdo che esista un nodo $x$ che trasgredisca questa proprietà ($x$.\emph{ft} $= i$). Supponiamo che $v_1, \dots, v_n \in [v]$ (in ordine decrescente di \emph{finishing-time}), e che $v_n$.\emph{ft} $= i-1$. Inoltre supponiamo che ci siano altri nodi in $[v]$, tutti con \emph{finishing-time} minore di $v_n$.\emph{ft}. Chiaramente deve valere $\langle v_n,x \rangle \in E^{-1}$, o equivalentemente $\langle x,v_n \rangle \in E$. Allora $x \in [v]$ per l'Osservazione \ref{obs:rank_img_scc}.
\end{proof2}

Per comodità, nel seguito adotteremo la seguente notazione:
\begin{align*}
    [v].\emph{ft}^* &\coloneqq \max_{x \in [v]} x.\emph{ft}\\
    [v].\emph{ft}_* &\coloneqq \min_{x \in [v]} x.\emph{ft}
\end{align*}

Deduciamo il seguente risultato:
\begin{proposition}
    \label{prop:rank_independent}
    $\rankfunc(v)$ dipende esclusivamente dal rango dei nodi $u$ che soddisfano la seguente proprietà:
    \begin{gather}
        u \in [v] \lor u.ft > [v].ft^* \label{rank_independency_formula}
    \end{gather}
\end{proposition}
\begin{proof2}
    Sia $x$  un nodo per cui non vale nessuna delle due condizioni, quindi $u$.\emph{ft} $< [v].\emph{ft}_*$. Allora dai nodi di $[v]$ non è possibile raggiungere $u$ (nemmeno con cammini più lunghi di un arco): se per assurdo fosse possibile, per il Corollario \ref{cor:scc_minore_inglobamento} si avrebbe $u \in [v]$; allora, per la definizione di \rankfunc, si ha la tesi.
\end{proof2}
Abbiamo spiegato perchè risulta conveniente operare la DFS su $G$ in ordine decrescente di \emph{finishing-time}: ad ogni passo saranno già disponibili le uniche informazioni necessarie, cioè il rango dei nodi con \emph{finishing-time} maggiore.

A questo punto abbiamo tutti gli ingredienti necessari per dimostrare il teorema che avevamo lasciato in sospeso, relativo alla correttezza dell'Algoritmo \ref{alg:rank}:
\begin{proof2}[Teorema \ref{theo:rank_correct}\,]
    Sia $v$ un nodo. Se $v$.\emph{rank} $= 0$, allora $v$ è un pozzo.

    Se $v$.\emph{rank} $\neq 0$, poichè il rango viene calcolato come da definizione, è sufficiente dimostrare che al momento della visita di $E(v)$ si ha $u$.\emph{rank} $= \rankfunc(u), \,\forall u \in E(v)$. Chiaramente gli unici nodi problematici sono quelli di colore grigio al momento della visita, in quanto l'esplorazione della loro immagine non è ancora terminata. Questi nodi stanno in $[v]$, e per il Corollario \ref{cor:no_buchi_scc} non ci sono nodi esterni alla SCC nell'intervallo, se si considera il \emph{finishing-time}. Ma allora tutti i nodi il cui rango è rilevante sono già stati visitati (Proposizione \ref{prop:rank_independent}).
\end{proof2}

\subsubsection{L'algoritmo}
Per comodità di notazione poniamo $\mathbb{N}^* \coloneqq \mathbb{N} \cup \{-\infty, 0\}$. Presentiamo e commentiamo lo pseudocodice dell'Algoritmo di Dovier-Piazza-Policriti:

\begin{algorithm}[H]
    \label{alg:fba}
    \KwData{$G = (V,E)$}
    \caption{Algoritmo di Dovier-Piazza-Policriti}
    \SetKwProg{Fn}{function}{:}{end}
    \Fn{\funcname{collapse}($G = (V,E), B \subseteq V$)}{
        Sia $u \in B$ scelto casualmente\;
        \ForEach{$v \in B - \{u\}$}{
            \ForEach{$z \in E(v)$}{
                $E = (E - \{\langle v,z\rangle\}) \cup \{\langle u,z\rangle\}$\;
            }
            \ForEach{$z \in E^{-1}(v)$}{
                $E = (E - \{\langle z,v\rangle\}) \cup \{\langle z,u\rangle\}$\;
            }
            $V = V - \{v\}$\;
        }
        \Return{$u$}\;
    }
    \Fn{\funcname{split2}($G = (V,E), P, u, \widehat{B}$)} {
        \ForEach{$B \in P \mid B \in \widehat{B}$}{
            $P = P - \{B\} \cup \{\{v \in B \mid \langle v, u\rangle \in E\},$\\
                $\qquad \{v \in B \mid \langle v, u \rangle \not\in E\}\}$\;
        }
    }
    \Begin{
        \funcname{compute-rank}($G$)\;
        $\rho \coloneqq \max\{\rankfunc(n) \mid n \in V\}$\;
        $B_k \coloneqq \{n \in V \mid \rankfunc(n) = k\}, \,\, k \in \mathbb{N}^*$\;
        $P \coloneqq \{B_i \mid i \in \mathbb{N}^*\}$\; \label{alg:partizione_rudimentale}

        \tcp*[h]{$u$ è il nodo di $B_{-\infty}$ preservato da \funcname{collapse}.}\\
        $u  \coloneqq$ \funcname{collapse}($G,B_{-\infty}$)\; \label{alg:impunemente_omega}
        \funcname{split2}($G, P, u, \bigcup_{i=0}^\rho B_i$)\; \label{alg:split_post_omega}

        \ForEach{$i = 0, \dots, \rho$}{
            \label{alg:fba_principal_loop}
            \tcp*[h]{Selezioniamo i blocchi di $P$ aventi rango $i$.}\\
            $D_i \coloneqq \{B \in P \mid B \subseteq B_i\}$\;
            \tcp*[h]{Il sottografo di $G$ dei nodi di rango $i$.}\\
            $G_i = (B_i, (B_i \times B_i) \cap E)$\;
            \tcp*[h]{$D_i$ è una partizione di $G_i$.}\\
            $D_i =$ \funcname{Paige-Tarjan}($G_i, D_i$)\;

            \ForEach{$B \in D_i$}{
                $u \coloneqq$ \funcname{collapse}($G,B$)\; \label{alg:collapse_rank}
                \funcname{split2}($G, P, u, \bigcup_{j=i+1}^\rho B_j$)\;
            }
        }
    }
\end{algorithm}

La funzione \funcname{collapse} rimuove dal grafo tutti i nodi all'interno di un blocco, ad eccezione di un nodo scelto in modo casuale; dopodichè sostituisce il nodo mantenuto in ogni arco incidente ad uno dei nodi rimossi. Viene utilizzata per ridurre ad un solo vertice un blocco per cui si è già stabilito che tutti i nodi sono bisimili.

La funzione \funcname{split2} è analoga alla funzione \splitfunc, ma consente di specificare i blocchi che possono essere modificati, solo quelli in $\widehat{B}$, e di ignorare quelli non appartenenti a $\widehat{B}$; inoltre lo \splitfunc viene effettuato rispetto ad un unico vertice, infatti ogni $B \in \widehat{B}$ viene rimpiazzato in $P$ da due blocchi $\{v \in B \mid \langle v, u\rangle \in E\}$ e $\{v \in B \mid \langle v, u \rangle \not\in E\}\}$, che sono rispettivamente la notazione non insiemistica degli insiemi $B \cap E^{-1}(\{u\}), B - E^{-1}(\{u\})$. Ci si aspetta che $u$ sia il nodo ``soppravvissuto'' da una chiamata a \funcname{collapse}.

L'algoritmo inizia con il calcolo del rango dei nodi del grafo. Alla Riga \ref{alg:partizione_rudimentale} viene creata una partizione iniziale, da rifinire, i cui blocchi sono composti dai nodi aventi lo stesso rango. Per il Teorema \ref{theo:bisi_rank} la \rscp sarà sicuramente una rifinitura di questa partizione.

Per la Proposizione \ref{prop:omega_rank} possiamo impunemente considerare bisimili tutti i nodi di rango $-\infty$. Per questo motivo alla Riga \ref{alg:impunemente_omega} viene collassato il blocco $B_\infty$. Si noti che un'assunzione del genere non è valida per altri valori del rango, in quanto nodi non bisimili possono avere lo stesso rango. A questo punto è necessario aggiornare la partizione e dividere ogni blocco che non rispetta la condizione di stabilità (Riga \ref{alg:split_post_omega}).

In seguito, per ogni rango a partire 0, si considerano i blocchi di rango $i$ e si isola il sottografo contenente solamente nodi composti dai nodi che vi sono contenuti. Si applica l'algoritmo di Paige-Tarjan a questo sottografo, al fine di calcolarne la \rscp. I blocchi di questa nuova partizione vengono collassati, ed analogamente a quanto fatto in precedenza si impone ai blocchi di rango superiore la condizione di stabilità. Si nodi che ad ogni iterazione il numero di blocchi e nodi influenzati dalla chiamata a \funcname{split2} si riduce, coerentemente con quanto abbiamo anticipato nella Sezione \ref{sec:fba_preliminari}.

\subsubsection{Analisi}
Come abbiamo fatto per gli algoritmi di Hopcroft e Paige-Tarjan, verifichiamo la correttezza e la complessità dell'Algoritmo \ref{alg:fba}.
\begin{theorem}
    Due nodi $m,n$ vengono collassati in un unico nodo durante l'esecuzione dell'Algoritmo \ref{alg:fba} se e solo se sono bisimili.
\end{theorem}
\begin{proof2}
    Siano $m,n \in V \mid m \equiv n$. Per il Teorema \ref{theo:bisi_rank} $m,n$ devono necessariamente stare all'interno dello stesso blocco nella partizione iniziale creata alla Riga \ref{alg:partizione_rudimentale}. Dimostriamo ora che durante l'esecuzione dell'algoritmo $m,n$ verranno collassati in un unico nodo.

    Se $\rankfunc(m) = \rankfunc(n) = -\infty$ il blocco contenente questi due nodi viene collassato alla Riga \ref{alg:impunemente_omega}.

    Per i valori positivi del rango, procediamo per induzione. Se $\rankfunc(m) = \rankfunc(n) = 0$ inizialmente appartengono entrambi a $B_0$. Sicuramente non vengono divisi da \funcname{split2} alla Riga \ref{alg:split_post_omega}, e l'intero blocco $B_0$ viene collassato alla Riga \ref{alg:collapse_rank}. Se $\rankfunc(m) = \rankfunc(n) = k > 0$, poichè abbiamo supposto che $m \equiv n$, si ha che $\langle m, m' \rangle \in E \implies \exists n' \mid \langle n, n' \rangle \in E \land \langle m', n' \rangle \in E$ (e lo stesso vale partendo da $n$). Questo significa che l'esistenza di un arco uscente da uno dei due nodi implica l'esistenza di un altro arco, uscente dall'altro nodo, e che le due destinazioni sono bisimili. Ma allora $m,n$ non possono essere stati divisi nei passaggi precedenti, perchè per l'ipotesi induttiva i nodi bisimili tra loro di rango inferiore a $k$ appartengono allo stesso blocco all'inizio dell'iterazione $i$-esima del ciclo alla Riga \ref{alg:fba_principal_loop}. Non c'è ancora stata alcuna divisione di blocchi indotta da nodi di rango uguale a $k$ (gli unici verso cui un nodo di rango $k$ può avere un arco, oltre ai nodi di rango strettamente minore di $k$), per cui $m,n$ sono necessariamente nello stesso blocco all'inizio dell'iterazione $i$-esima. Per la correttezza dell'Algoritmo di Paige-Tarjan, $m,n$ appartengono ancora allo stesso blocco quando questo viene collassato alla Riga \ref{alg:collapse_rank}. In modo analogo si dimostra che se due nodi appartengono allo stesso blocco quando questo viene collassato, allora sono necessariamente bisimili.
\end{proof2}
Quest'ultimo risultato dimostra chiaramente la correttezza dell'algoritmo di Dovier-Piazza-Policriti. Prima di valutare la complessità, discutiamo il costo delle funzioni ausiliarie:
\begin{observation}
    Un'implementazione efficiente congiunta delle funzioni \funcname{collapse} e \funcname{split2} ha complessità $\Theta(|E^{-1}(B)|)$, dove $B$ è il blocco su cui sono chiamate una dopo l'altra.
\end{observation}
\begin{proof2}
    \accente sufficiente creare inizialmente l'insieme $E^{-1}(\{v\}) \, \forall v \in V$, cioè la contro-immagine di $v$ rispetto alla relazione binaria $E$. Per collassare il blocco $B$ e ricalcolare la partizione, cioè per eseguire le due seguenti righe di pseudocodice:
    \begin{align*}
        &u \coloneqq \textup{collapse}(G,B);\\
        &\funcname{split2}(G, P, u, \bigcup_{j=i+1}^\rho B_j);
    \end{align*}
    è sufficiente visitare in modo esaustivo l'insieme $E^{-1}(B)$ e per ogni blocco $C$ della partizione iniziale (tra quelli contenuti in $\bigcup_{j=i+1}^\rho B_j$) raggiunto da questa visita rimuovere da $C$ i nodi che rientrano nella contro-immagine di $B$, ed inserirli in un nuovo blocco.
\end{proof2}
\begin{theorem}
    L'Algoritmo \ref{alg:fba} ha complessità $O(|E| \log |V|)$.
\end{theorem}
\begin{proof2}
    Il rango dei nodi del grafo si calcola con l'Algoritmo \ref{alg:rank}, che ha complessità $O(|V|+|E|)$. Per l'osservazione precedente le operazioni alle Righe \ref{alg:impunemente_omega}, \ref{alg:split_post_omega} hanno un'implementazione lineare.

    Il ciclo alla Riga \ref{alg:fba_principal_loop} è composto da tre elementi:
    \begin{enumerate}
        \item La creazione dell'insieme $D_i$ e del sottografo $\restr{G}{i}$;
        \item La chiamata all'Algoritmo PTA;
        \item Il ciclo sui blocchi appartenenti all'insieme $D_i$.
    \end{enumerate}
    Gli elementi di tipo 1 sono chiaramente dominati dagli elementi di tipo 2, in quanto determinare il sottografo $\restr{G}{i}$ ha complessità $O(|B_i| + |E \cap (B_i \times B_i)|)$, e lo stesso vale per gli elementi di tipo 3. La complessità degli elementi di tipo 2 invece è $O(|E \cap (B_i \times B_i)| \log |B_i|)$, come abbiamo già dimostrato (Teorema \ref{theo:pta_complex}).

    Ne segue che la complessità dell'algoritmo è:
    \begin{align*}
        T(V,E) &= O(|V| + |E|) + \sum_{i=0}^\rho O(|E \cap (B_i \times B_i)| \log |B_i|)\\
        &= O(|V| + |E|) +  O(\log |V|) \sum_{i=0}^\rho O(|E \cap (B_i \times B_i)|)\\
        &= O(|V| + |E|) +  O(\log |V| |E|)\\
        &= O(|E| \log |V|)
    \end{align*}
\end{proof2}

Come abbiamo anticipato sopra, la complessità identica all'algoritmo di Paige-Tarjan non rende l'algoritmo di Dovier-Piazza-Policriti poco interessante: quest'ultimo infatti dovrebbe risultare molto più performante qualora la partizione iniziale $P = \{B_k \mid k \in \mathbb{N}\}$ contenga molti blocchi di piccola dimensione. \accente quello che succede ad esempio con gli alberi bilanciati. \accente necessario comunque tenere in considerazione il calcolo preliminare del rango, che potrebbe rendere l'algoritmo più performante solo asintoticamente.
