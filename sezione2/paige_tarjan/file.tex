\subsection{Algoritmo di Paige-Tarjan}
\label{sec:pta}
In questa sezione esamineremo il più celebre procedimento per il calcolo della massima bisimulazione, l'algoritmo di Paige-Tarjan \cite{paigetarjan}. \accente stato il primo metodo efficiente per la risoluzione del problema nel caso generale (rammentiamo infatti che l'algoritmo di Hopcroft tratta un caso particolare), e migliora la complessità $O(|E||V|)$ dell'algoritmo di Kanellakis e Smolka, che non adotta particolari accorgimenti per la selezione dei blocchi da usare come \emph{splitter} (si ricordi che $E$ è l'insieme degli archi del grafo, e $V$ l'insieme dei nodi). Innanzitutto presenteremo e commenteremo lo pseudocodice, utilizzando la terminologia introdotta nella Sezione \ref{sec:base}; dopodichè dimostreremo la correttezza e la complessità.

\subsubsection{L'algoritmo}
L'algoritmo di Paige-Tarjan prende in input gli insiemi $V, E$ di nodi e degli archi, e la partizione iniziale $\mathfrak{V}$ da rifinire.
Inizialmente vengono definite due partizioni $Q,X$: la prima coincide con la partizione iniziale $\mathfrak{V}$, mentre la seconda è la partizione banale formata da un unico blocco contenente tutto $V$.

\begin{algorithm}[H]
    \label{alg:pt}
    \KwData{$V,E,\mathfrak{V}$}
    \caption{Algoritmo di Paige-Tarjan}
    \Begin{
        $Q \coloneqq \mathfrak{V}, X \coloneqq \{V\}$\;
        \tcp*[h]{I blocchi di $Q$ vengono rifiniti rispetto a $E^{-1}(V)$}\\
        $Q = \splitfunc(V,Q)$\; \label{alg:pta_rifi_iniziale}
        \While{$Q \neq X$}{
            Scelgo in modo casuale un blocco $S \in X \mid B \not\in Q$\;
            Scelgo un blocco $B \in X \mid B \subset S, |B| \leq |S|/2$\;
            $X = (X - \{S\}) \cup \{B, S-B\}$\;
            $Q = \splitfunc(S-B, \splitfunc(B,Q))$\;\label{alg:pt_doublesplit}
        }
        \Return{$Q$}\;
    }
\end{algorithm}

Alla Riga \ref{alg:pta_rifi_iniziale} viene imposta l'invariante che verrà poi mantenuta per tutta l'esecuzione dell'algoritmo: ogni blocco di $Q$ viene diviso in due blocchi (di cui uno eventualmente vuoto): nel primo vengono inseriti i nodi che sono sorgente di almeno un arco; nel secondo vengono inseriti i pozzi (cioè gli elementi dell'insieme $B - E^{-1}(V)$, dove $B$ è un blocco di $Q$). Dopo questa operazione la partizione $Q$ è stabile rispetto a $V$.

Il corpo del procedimento è formato da un ciclo che ripete le seguenti istruzioni finchè le partizioni $Q,X$ non coincidono:
\begin{enumerate}
    \item Viene scelto casualmente un blocco $S \in X$ non appartenente a $Q$;
    \item Viene scelto casualmente un blocco $B \in Q \mid B \subset S$, e la cui cardinalità sia inferiore a $|S|/2$;
    \item In $X$ il blocco $S$ viene sostituito da $\{B, S-B\}$;
    \item Si sostituisce la partizione $Q$ con \splitfunc$(S-B,$ \splitfunc$(B,Q))$.
\end{enumerate}

Quando $Q$ e $X$ coincidono, viene restituita la partizione $Q$.

\begin{observation}
    \accente sempre possibile trovare un blocco in $X$ per portare a termine il passaggio 1. Quando non è possibile l'algoritmo termina, perchè $Q = X$.
\end{observation}

\subsubsection{Analisi}
La correttezza dell'algoritmo di Paige-Tarjan si basa sul mantenimento di un'invariante, che enunciamo e dimostriamo formalmente:
\begin{lemma}
    \label{lem:pt_qx}
    L'Algoritmo \ref{alg:pt} ristabilisce le seguenti relazioni invarianti dopo ogni iterazione del ciclo principale:
    \begin{enumerate}
        \item $Q$ è stabile rispetto ad ogni blocco di $X$;
        \item $Q$ rifinisce $X$;
        \item $\rscp(\mathfrak{V},E)$ rifinisce $Q$.
    \end{enumerate}
\end{lemma}
\begin{proof2}
    In entrambi i casi procediamo per induzione sulle iterazioni:
    \begin{itemize}
        \item[1/2.] Prima della prima iterazione $Q$ è stabile rispetto a $X$ per l'operazione eseguita alla Riga \ref{alg:pta_rifi_iniziale}. Inoltre $Q$ rifinisce $X$ perchè quest'ultima è la partizione banale di $V$.

        Ora supponiamo che l'invariante sia valida prima di un'iterazione qualsiasi: durante l'iterazione successiva in $X$ viene sostituito il blocco $S$ con i blocchi $S-B, B$, mentre $Q$ viene modificato da sue applicazioni successive della funzione Split. \accente evidente che $Q$ deve essere stabile rispetto a $S-B, B$, ed anche a tutti gli altri blocchi di $X$ per l'ipotesi induttiva (si ricordi la Proposizione \ref{prop:rifi_stabile}). Inoltre, sempre per l'ipotesi induttiva, devono esistere (prima della modifica) dei blocchi $C_1, \dots, C_n \mid \bigcup_{i=1}^n C_i = S$. Dopo la modifica di $Q$ l'unione dei nuovi blocchi generati dall'applicazione dei due Split ai blocchi $\{C_1, \dots, C_n\} - B$ è $S-B$, mentre l'unione dei due nuovi blocchi (di cui uno eventualmente vuoto) generati dalla divisione di $B$ è chiaramente $B$. Per cui $Q$ rifinisce ancora $X$.
        \item[3.] La proprietà è vera banalmente prima della prima iterazione.

        Supponiamo che l'invariante sia valida prima dell'iterazione $i$-esima: durante l'iterazione successiva la
        partizione $Q$ viene rifinita utilizzando come \emph{splitter} i blocchi $B, S-B$. Poichè $Q$ rifinisce $X$, e $S$ è un blocco di $X$, $S$ è l'unione di alcuni blocchi di $Q$. Inoltre, essendo $B$ un blocco di $Q$ (con $B \subset S$), anche $S-B$ è unione di alcuni blocchi di $Q$. Allora, poichè per l'ipotesi induttiva $\rscp(\mathfrak{V},E)$ rifinisce $Q$:
        \begin{gather*}
            \rscp(\mathfrak{V},E) = \splitfunc(S-B, \splitfunc(B, \rscp(\mathfrak{V},E)))
        \end{gather*}
        Quindi, per il Teorema \ref{theo:split_properties} (monotonia di \splitfunc), $\rscp(\mathfrak{V},E)$ rifinisce la nuova partizione $Q' = \splitfunc(S-B, \splitfunc(B, Q))$.
    \end{itemize}
    \vspace*{-0.75cm}
\end{proof2}

Dal Lemma \ref{lem:pt_qx} possiamo dedurre direttamente il seguente risultato:
\begin{corollary}
    L'Algoritmo \ref{alg:pt} termina.
\end{corollary}
\begin{proof2}
    Ad ogni iterazione il numero di blocchi in $X$ aumenta di una unità. Quindi in $O(|V|)$ iterazioni si avrà $X = \{\{v\} \mid v \in V\}$ (se l'algoritmo non termina prima). Ma per il Lemma \ref{lem:pt_qx} $Q$ rifinisce $X$ prima di ogni iterazione, quindi deve essere necessariamente $Q = \{\{v\} \mid v \in V\} = X$.
\end{proof2}

A questo punto possiamo dimostrare in modo molto agevole la correttezza dell'algoritmo di Paige-Tarjan, utilizzando gli ingredienti proposti sopra:
\begin{proposition}
    Al termine dell'Algoritmo \ref{alg:pt} vale l'uguaglianza:
    \begin{gather*}
        Q = \rscp(E,\mathfrak{V}).
    \end{gather*}
\end{proposition}
\begin{proof2}
    Per il Lemma \ref{lem:pt_qx}, la partizione $Q$ è sempre stabile rispetto ad ogni blocco di $X$. Poichè la condizione di terminazione è $Q=X$, $Q$ è sempre stabile rispetto a $X$, ed in particolare al termine dell'algoritmo. Ma allora, sempre per il Lemma \ref{lem:pt_qx} (punto 3), al termine dell'algoritmo si ha $Q = \rscp(\mathfrak{V},E)$.
\end{proof2}

L'efficienza dell'algoritmo esposto all'inizio della sezione presente è garantita da un dettaglio implementativo che non abbiamo ancora reso esplicito. Introduciamo l'intuizione alla base tramite il seguente risultato \cite{paigetarjan}:

\begin{lemma}
    \label{lem:pt_lemma3}
    Valgono le seguenti uguaglianze insiemistiche per la doppia chiamata alla funzione $\splitfunc$ alla Riga \ref{alg:pt_doublesplit}:
    \begin{enumerate}
        \item $\splitfunc(B,Q)$ divide i blocchi $D \in Q$ in due blocchi $D_1 = D \cap E^{-1}(B), D_2 = D - D_1$. $D_1, D_2$ sono entrambi non vuoti se e solo se $D \cap E^{-1}(B) \neq \emptyset \land D - E^{-1}(B) \neq \emptyset$;
        \item $\splitfunc(S-B,\splitfunc(B,Q))$ divide i blocchi $D_1$ di $\splitfunc(B,Q)$ in due blocchi $D_{11} = D_1 \cap E^{-1}(S-B), D_{12} = D_1 - D_{11}$. $D_{11}, D_{12}$ sono entrambi non vuoti se e solo se $D_1 \cap E^{-1}(S-B) \neq \emptyset \land D_1 - E^{-1}(S-B) \neq \emptyset$;
        \item $\splitfunc(S-B,\splitfunc(B,Q))$ non modifica i blocchi di tipo $D_2$;
        \item $D_{12} = D_1 \cap (E^{-1}(B) - E^{-1}(S-B))$.
    \end{enumerate}
\end{lemma}
\begin{proof2}
    La dimostrazione è banale per i primi due punti (è una conseguenza diretta della Definizione \ref{def:funz_split}).
    \begin{enumerate}
        \item[3.] Supponiamo che un blocco $D \in Q$ sia stato diviso in due blocchi non vuoti $D_1, D_2$ da $\splitfunc(B,Q)$. Allora, per il punto 1, $D \cap E^{-1}(B) \neq \emptyset$. Quindi, per la stabilità di $Q$ rispetto a $(E,S)$, deve valere $D \subseteq E^{-1}(S)$. Allora $D_2 \subset D \subseteq E^{-1}(S)$. Per definizione $D_2$ e $E^{-1}(B)$ sono disgiunti, quindi $D_2 \subseteq E^{-1}(S-B)$;
        \item[4.] Evidente nella Figura \ref{fig:pt_lemma_insiemi}.
    \end{enumerate}
    \vspace*{-0.75cm}
\end{proof2}

\begin{figure}[H]
    \centering

    \begin{venndiagram3sets}[labelA=$D_1$, labelB=$E^{-1}(B)$, labelC=$E^{-1}(S-B)$, showframe=false, labelOnlyAB=$D_{12}$]
        \fillACapBNotC
    \end{venndiagram3sets}

    \caption{Punto 4. del Lemma \ref{lem:pt_lemma3}}
    \label{fig:pt_lemma_insiemi}
\end{figure}

Di conseguenza abbiamo determinato la tipologia di blocco che può essere generata, a partire da un blocco di $Q$, dalla Riga \ref{alg:pt_doublesplit} dell'algoritmo di Paige-Tarjan. Possiamo sfruttare questa informazione per implementare il procedimento in modo efficiente.

\begin{observation}
    Con opportuni accorgimenti e strutture dati \cite{paigetarjan} è possibile implementare una funzione che effettua l'operazione $\splitfunc(S-B, \splitfunc(B,Q))$ con complessità $O(|B| + \sum_{y \in B} |E^{-1}(\{y\})|)$.
\end{observation}

Da questa implementazione deduciamo il seguente upper-bound sulla complessità dell'Algoritmo di Paige-Tarjan:
\begin{theorem}
    La complessità dell'Algoritmo \ref{alg:pt} è $O(|E|\log |V|)$.
\end{theorem}
\begin{proof2}
    Chiaramente la complessità è:
    \begin{align*}
        \displaystyle &\sum_{\substack{B \mid B \text{ è usato}\\\text{come \emph{splitter}}}} O(|B| + \sum_{y \in B} |E^{-1}(\{y\})|)\\
        &= \sum_{\substack{B \mid B \text{ è usato}\\\text{come \emph{splitter}}}} O(|B|) + \sum_{\substack{B \mid B \text{ è usato}\\\text{come \emph{splitter}}}} \sum_{y \in B} |E^{-1}(\{y\})|)
    \end{align*}

    \begin{observation}
        Ogni elemento $x \in S$ può apparire al più $\log_2 (|V| + 1)$ volte in un blocco usato come \emph{splitter}, visto che ad ogni nuova iterazione un eventuale blocco contentente questo stesso $x$ ha cardinalità dimezzata.
    \end{observation}

    Allora:
    \begin{align*}
        \sum_{\substack{B \mid B \text{ è usato}\\\text{come \emph{splitter}}}} \sum_{y \in B} |E^{-1}(\{y\})|) &\leq \log_2 (|V|+1) \sum_{x \in V} |E^{-1}(\{x\})|\\
        &= \log_2 (|V|+1) |E|\\
        &= O(|E|\log |V|)
    \end{align*}

    Inoltre:
    \begin{align*}
        \sum_{\substack{B \mid B \text{ è usato}\\\text{come \emph{splitter}}}} O(|B|) &\leq \sum_{x \in V} \log_{2} (|V|+1)\\
        &= |V|\log_{2} (|V|+1)\\
        &= O(|V|\log |V|)
    \end{align*}

    \accente lecito considerare $|V| = O(|E|)$, per cui la complessità è $O(|E| \log |V|)$.
\end{proof2}
